%\documentclass[article]{jss}
\documentclass[nojss]{jss}

%% -- LaTeX packages and custom commands ---------------------------------------

\usepackage{orcidlink,thumbpdf,lmodern}
\usepackage{mathtools,amssymb,amsthm,amsmath, bm}
\usepackage{adjustbox}
\usepackage{longtable}
\usepackage{booktabs} % for toprule
\usepackage{bookmark}
\usepackage{textcomp} % for registered symbol (in biblio)
\usepackage{multirow}
\usepackage{arydshln} % for dashed line in tables
\usepackage[flushleft]{threeparttable} % for table note
\usepackage{pdflscape}
\usepackage{geometry}
%\usepackage{fancyhdr}
%\usepackage{tabularx, colortbl, array}
\usepackage{scrlayer-scrpage}

%\setlength{\arrayrulewidth}{0.5mm}
%\renewcommand{\arraystretch}{1.5}
%\setlength{\tabcolsep}{18pt}

\DeclareNewLayer[
  background,
  rightmargin,
  contents={%
    \parbox[b][\layerheight][c]{\dimexpr\footskip+\footheight\relax}{%
      \hfill\rotatebox{90}{\pagemark}}}
]{lscape.foot}

\DeclareNewLayer[
  background,
  textarea,
  addvoffset=-\dimexpr\headsep+\headheight\relax, % Adjust vertical offset to center the header
  addhoffset=-\dimexpr\footskip+\footheight\relax, % Adjust horizontal offset to center the header
  width=\dimexpr\headsep+\headheight+\textheight+\footskip+\footheight\relax, % Adjust width to center the header
  height=\dimexpr\textwidth\relax, % Swap width and height to match the landscape orientation
  contents=\rotatebox{90}{\parbox{\layerheight}{\centering\headmark}}
]{lscape.head}

\DeclareNewPageStyleByLayers{lscape}{lscape.foot,lscape.head}

% \VignetteIndexEntry{Introduction to the package sfaR}

\newcommand{\sfar}{\pkg{sfaR}}



%% -- Article metainformation (author, title, ...) -----------------------------

\author{K Herv\'e Dakpo~\orcidlink{0000-0002-6114-7896}\\Universit\'e 
Paris-Saclay INRAE \\ AgroParisTech, PSAE\\ Palaiseau, F-91120, France \AND
Yann Desjeux ~ \\INRAE, Bordeaux School of Economics, \\ University Bordeaux, 
Pessac, France \AND Arne Henningsen ~ \\ Dept. of Food and Resource Economics,\\ 
University of Copenhagen, Frederiksberg C, Denmark \AND Laure Latruffe ~ \\
INRAE, Bordeaux School of Economics, \\ University Bordeaux, 
Pessac, France}
\Plainauthor{K Herv\'e Dakpo, Yann Desjeux, Arne Henningsen, Laure Latruffe}

\title{Stochastic Frontier Analysis Using \pkg{sfaR}}
\Plaintitle{Stochastic Frontier Analysis Using sfaR}
\Shorttitle{\pkg{sfaR}: SFA using R}

%% - \Abstract{} almost as usual
\Abstract{
  This article gives an overview of the capabilities of the \pkg{sfaR} package
  in estimating stochastic frontier (SF) models. It extends the commonly used
  \pkg{frontier} package by including new models for cross-section and panel 
  data. Many of the recent advances in the field like the latent class SF models 
  and a few of its variants (zero inefficiency SF, contaminated noise SF, and
  multimodal inefficiency SF), the metafrontier SF, and the sample selection SF 
  are included in the \pkg{sfaR} package. For many of the models, several 
  distributions are possible for the one-sided error term (e.g., half-normal, 
  truncated normal, exponential, rayleigh, generalized exponential, truncated
  skewed-laplace, uniform, gamma, weibull, and log-normal distributions. Another 
  advantage of the \pkg{sfaR} package is to allow many possibilities (eleven) in 
  terms of optimization algorithm. A description of the main functions in the 
  \pkg{sfaR} package, in addition with illustration using real data are provided.
}

%% - \Keywords{} with LaTeX markup, at least one required
%% - \Plainkeywords{} without LaTeX markup (if necessary)
%% - Should be comma-separated and in sentence case.
\Keywords{standard stochastic frontiers, latent class stochastic frontiers, 
metafrontiers, sample selection stochastic frontier, zero inefficiency stochastic
frontiers, cross-section data, panel data, \proglang{R}}
\Plainkeywords{standard stochastic frontiers, latent class stochastic frontiers, 
metafrontiers, sample selection stochastic frontier, zero inefficiency stochastic
frontiers, cross-section data, panel data, R}

%% - \Address{} of at least one author
%% - May contain multiple affiliations for each author
%%   (in extra lines, separated by \emph{and}\\).
%% - May contain multiple authors for the same affiliation
%%   (in the same first line, separated by comma).
\Address{
  K Herv\'e Dakpo\\
  Journal of Statistical Software\\
  Universit\'e Paris-Saclay INRAE\\
  AgroParisTech, PSAE\\
  Palaiseau, F-91120, France\\
  E-mail: \email{k-herve.dakpo@inrae.fr}\\
  URL: \url{https://www6.versailles-grignon.inrae.fr/psae_eng/PersonalPages2/Herve-Dakpo}
}

 
\usepackage{Sweave}
\begin{document}
\Sconcordance{concordance:sfaR_handout.tex:sfaR_handout.Rnw:1 49 1 1 5 59 1 %
1 0 818 1 1 5 4 0 1 5 3 0 1 5 3 0 1 6 4 0 1 5 6 0 1 2 9 1 1 32 %
1 2 11 1 1 2 80 0 1 2 3 1 1 2 1 0 1 3 5 0 1 2 5 1 1 4 48 0 1 2 %
8 1 1 32 15 0 1 2 3 1 1 5 4 0 1 4 3 0 1 4 8 0 1 4 8 0 1 4 8 0 %
1 4 3 0 1 4 6 0 1 2 430 1}


%% -- Introduction -------------------------------------------------------------

%% - In principle "as usual".
%% - But should typically have some discussion of both _software_ and _methods_.
%% - Use \proglang{}, \pkg{}, and \code{} markup throughout the manuscript.
%% - If such markup is in (sub)section titles, a plain text version has to be
%%   added as well.
%% - All software mentioned should be properly \cite-d.
%% - All abbreviations should be introduced.
%% - Unless the expansions of abbreviations are proper names (like "Journal
%%   of Statistical Software" above) they should be in sentence case (like
%%   "generalized linear models" below).

\section[Introduction]{Introduction} \label{sec:intro}

Since its inception over 40 years ago by \citet{aig77, mee77, batt77}, the 
stochastic frontier analysis (SFA) has been used as a proeminent performance 
benchmarking tool in many aspects of decision making units (DMUs). Since then 
several models have been developed to tackle different situations (data type, 
technological heterogeneity, endogeneity, $\cdots$). Overall, SFA can be used 
to evaluate different performance indicators for DMUs (technical and economic 
efficiencies, productivity, $\cdots$). By estimating the efficiency scores of 
DMUs, we can compare them, find out who is under-performing, and learn from the 
best practices. This can help make better policies or subsidies to improve the 
efficiency of both private and public sectors \citep{par14}. 

SFA has been used in a variety of areas and for different purposes. For 
instance, SFA has been used to measure productivity growth and its sources 
\citep{orea02}, or to examine the effects of regulation and competition on 
performance \citep{myd20, kang02}, to evaluate eco-efficiency of firms 
\citep{mada23}, to analyze the impact of environmental factors \citep{li19}. 
SFA has been use in agriculture \citep{latruffe23}, banking \citep{badu21}, 
health care \citep{ji20}, manufacturing sector \citep{boy19}, $\cdots$. 

The scope of SFA is so huge that it spreads from pure research objectives to 
practitionners and policy makers. As such, the \pkg{sfaR} package intends to 
provide a platform for the basics but also for some of the recent advances in 
this field of performance benchmarking using the software 
\proglang{R}.\footnote{SFA has been implemented in other sofware like 
\proglang{Limdep}, \proglang{Stata}, \proglang{Ox}, $\cdots$}. In \proglang{R}, 
the package \pkg{frontier} is without any doubt the biggest attempt in 
an open software \citep{frontier}.\footnote{The package \pkg{frontier} uses the 
Fortran source codes of \proglang{Frontier 4.1} originally developed by Tim 
Coelli.} However, this package is only limited to a few distributions for the 
inefficiency term and a few panel models. For instance, only the half-normal and
truncated normal distributions are possible in \pkg{frontier}. Regading the panel
models, \pkg{frontier} implements the time-invariant inefficiency model by 
\citet{pl81} and the time-varying efficiency model by \citet{batco92}. \pkg{sfaR} 
goes a step further and provide more alternative in the time-varying inefficiency
models \citep{kum90, batco92, lees93, cue00, cue02, feng09, kw05, al06}.

Recently, the package \pkg{npsf} by \citet{npsf} has extended the \pkg{frontier}
package to new panel models (e.g., the generalized true random effects as 
discussed in \citet{colomb14, klh14, tk14}).\footnote{Many other packages can be
found in the literature. For instance, the \pkg{Benchmarking} package contains 
a very basic SFA model \citep{bench}, \pkg{sfadv} provides one methodology to 
deal with endogeneity in SFA \citep{sfadv}, \pkg{semsfa}, and \pkg{dsfa} offer semi- and 
non-parametric SFA models \citep{semsfa, dsfa}, and \pkg{ssfa} allows for spatial 
stochastic frontiers \citep{ssfa}. In addition to these packages, several R codes can be found
online to conduct some SFA \citep{si20, si22, nguyen22}.} Still, many 
of the recent developments are still lacking, especially regarding technological
heterogeneity. For instance, the latent class stochastic frontier models, and 
many of its variants are not covered by any of aforementioned packages. For instance,
\pkg{sfaR} includes the latent class SF, the zero inefficiency SF, the 
contaminated noise SF, and the multi-modal inefficiency SF \citep{orea04, 
greene05, kum13, rho15, wheat19}. The same is true for the metafrontier 
\citep{batt04, huang14, am17}, and the sample selection bias 
\citep{greene10}. 

With \pkg{sfaR} we do propose not an exhaustive universe of the 
stochastic frontier (SF) models, but maybe a first step towards this direction. 
As we expect the package to grow and even include other developers extension. As of
now, the package, in summary, can handle cross-section but also panel data depending on the 
models. It offers more flexibility in terms of the distributions for the 
inefficiency term, but also the possibility to use several optimization 
algorithms. Regarding the models, 296 likelihood functions are currently 
implemented and grouped into ten main functions. This article summarizes the main information 
about each group of models implemented and illustrate them using real life data.
The structure of the paper is as follows. Section \ref{sec:modelstandar} gives 
a brief overview of the standard SF models in the literature for cross-section
and panel data. Section \ref{sec:modelcm} describes the latent class SF in
addition to its variants. Finally, section \ref{sec:other} presents the other 
SF models present the \pkg{sfaR}, namely the metafrontiers and the sample 
selection SF. Practically, section \ref{sec:prespackage} draws a skeleton
of the \pkg{sfaR} package focusing on the main functions and some postestimation
routines, and section \ref{sec:illus} provides some empirical applications. The
last section offers some discussions and conclusions.

\section{Standard models for cross-section and panel data} \label{sec:modelstandar}

\subsection{Standard cross-sectional SFA models}

\subsubsection{Densities}

Let's consider the standard production function \footnote{Other specifications 
for the production technology can also be used, e.g. distance functions, profit,
cost or revenue functions, input/output requirement functions.} 
%
\begin{equation}\label{eq:1}
y_i = \mathbf{x_i'}\bm{\beta} + v_i - Su_i \quad \text{with} \quad i = 1, \cdots, N
\end{equation}
%
where $S = 1$ for production function and $S = -1$ for cost function. $v$ is the 
two-sided (idiosyncratic) error, which is generally assumed to follow a normal distribution 
$\mathcal{N}(0, \sigma_v^2)$. 

The spearhead for the SFA is the assumption regarding the distibution of $u$. 
In the \pkg{sfaR} package, ten different distributions with $u \geq 0$ are 
possible for the standard model. Table~\ref{table:dens} presents the density 
function of all the distributions allowed by \pkg{sfaR}.\footnote{Moments associated
with each distribution can be found in Table \ref{table:momentdesn} in the 
Appendix.} For each of the distribution, the parameters can be obtained by 
maximizing the log-likelihood giving the convolution of $\epsilon_i=v_i - Su_i$. 

The most commonly used distributions are the one-parameter half-normal 
\citep{aig77}, and exponential distributions \citep{mee77}. Their large success 
is attributed to the simplicity of their density function and their availability 
in most statistical software (especially for the half-normal distribution). 
However, these distributions assumed a zero mode for the inefficiency implying 
that most DMUs will be deemed efficient. Two-parameter distributions like the 
truncated normal \citep{ste80, kum87}, and the gamma \citep{ste80, beckers87, greene90, greene03}
were later introduced to shift away from the zero mode distributions. 
A particular feature of these two distributions is that, under specific 
conditions they nest the half-normal and the exponential distributions, 
respectively. In the case of the gamma distribution, the density corresponding 
to $\epsilon_i$ does not have a close form, and \citet{greene03} suggested the 
use of maximum simulated likelihood (MSL). If the previous four distributions 
are the standard in the stochastic frontier framework, other 
distributions are also introduced. For instance, the rayleigh \citep{oliv14, 
hajar15, wang20}, and the generalized exponential \citep{papa21} are 
one-parameter distributions, which have non-zero mode. Another alternative 
two-parameter distribution is the log-normal \citep{wang20}, 
which can be a good choice for certain type of datasets \citep{migon01, rama12}.

A common feature of all the previous distributions is their positive skewness, 
which supposed a negative third moment for the distribution of 
$\epsilon_i$ when $S=1$. However, in practice, the third moment of 
$\epsilon_i$ can be positive \citep{sw09}. This generate a type I error, which 
might prevent the identification of the inefficiency distribution parameters 
\citep{olson80}. According to \citet{wald82}, the "wrong" skewness implies that 
the maximum likelihood and the OLS estimates of equation~\ref{eq:1} will have 
the same slopes and there are no inefficiencies. A long debate has then emerged 
around the "wrong" skewness. An advantage of the weibull and 
the truncated skewed-laplace distributions is to allow positive 
or negative skeweness \citep{tsionas07, wang12}, and therefore identifying inefficiencies even when the OLS
residuals has the "wrong" skewness. On another hand, the uniform distribution 
\citet{li96, nguyen10, lee14}, which has a zero skewness, has been introduced to  
handle situations where the inefficiency distribution is symmetric around the 
mean. In this latter case, the OLS residuals will also be symmetric, and based on 
the previous distributions, no inefficiencies can be measured. Another advantage
of the uniform distribution is to set a threshold on the inefficiency, which can
provide useful economic information \citep{lee14}.

\textbf{Do distributional assumptions even matter?} If as underlined by 
\citet[p.~16]{kum20} the "choice of $u$ is often driven through available statistical 
software to implement the method rather than an underlying theoretical link 
between a model of productive inefficiency and the exact shape of the 
corresponding distribution", the \pkg{sfaR} package will fill in some gaps and 
allow practionners to rigorously check differences in their results.\footnote{See 
also \citet[p.~214-216]{par14} for more discussion on the matter of the choice of $u$ distribution.}
It is worth mentioning that if one is only interested on the features of the 
frontier, as long as inefficiency is not conditional on some variables, the OLS
results should be sufficient.


%\begin{landscape}
\begin{table}[t]
%\setlength{\arrayrulewidth}{.01em}
\renewcommand{\arraystretch}{1.3}
\centering
%\begin{adjustbox}{max width=1\textwidth}
\begin{tabular}{@{}cc@{}}
\toprule
Distributions & Densities \\
\midrule
Half-Normal & $f(u) = \frac{2}{\sigma_u}\phi\left(\frac{u_i}{
\sigma_u}\right)$ \\[1em]
%\hdashline
Truncated-Normal & $f(u) = \frac{\frac{1}{\sigma_u}\phi\left(
\frac{u_i-\mu}{\sigma_u}\right)}{1 - \Phi\left(-\frac{\mu}{\sigma_u}\right)}$ \\[1em]
%\hdashline
Exponential & $f(u) = \frac{1}{\sigma_u}\exp{\left(
-\frac{u}{\sigma_u}\right)}$ \\[1em]
%\hdashline
Rayleigh & $f(u) = u\exp{\left(-\frac{u^2}{2\sigma_u^2}\right)}/
\sigma_u^2$\\[1em]
%\hdashline
Gamma & $f(u) = \frac{\sigma_u^{-P}}{\Gamma\left(P\right)}\exp{
\left(-u/\sigma_u\right)}u^{P-1}$ with $P > 0$\\[1em]
%\hdashline
Log-Normal & $f(u) = \frac{1}{u\sigma_u}\phi\left(\frac{
\ln{u}-\mu}{\sigma_u}\right)$\\[1em]
%\hdashline
Weibull & $f(u) = \frac{k}{\sigma_u}\left(\frac{u}{\sigma_u}
\right)^{k-1}\exp{\left(-\left(u/\sigma_u\right)^{k}\right)}$ with $k > 0$\\[1em]
%\hdashline
Generalized Exponential & $f(u) = \frac{2}{\sigma_u}\left[1-\exp{
\left(-u/\sigma_u\right)}\right]\exp{\left(-u/\sigma_u\right)}$ \\[1em]
%\hdashline
Truncated Skewed-Laplace & $f(u) =\frac{1+\lambda}{\sigma_u\left(
2\lambda+1\right)}\left\{2\exp{\left(-\frac{u}{\sigma_u}\right)}-
\exp{\left(-\frac{\left(1+\lambda\right)u}{\sigma_u}\right)}\right\}$ with 
$\lambda > 0$\\[1em]
%\hdashline
Uniform & $f(u) = \frac{1}{\theta}$ with $u \in [0, \theta]$\\[1em]
\bottomrule
\end{tabular}
%\end{adjustbox}
\caption{List of distributions for $u$}
\label{table:dens}
\end{table}
%\end{landscape}


The convolution of $\epsilon$ associated with each of the distribution in 
Table~\ref{table:dens} are presented in Table~\ref{table:mlsfcross} in the 
Appendix.\footnote{Proof of the derivation of these densities can be found in 
the corresponding literature or from the authors upon request.} For estimation 
purposes, the variances of $u$ and $v$ are parameterized as follows:
$\sigma_u^2 = \exp{\left(W_u\right)}$ and $\sigma_v^2 = \exp{\left(W_v\right)}$.
In the case of the uniform distribution we have $\theta = \sqrt{12\sigma_u^2}$.
Heteroscedasticity is handled in both error terms, and the variances are again 
parameterized as
%
\begin{align*}
\sigma_{u,i}^2 &=\exp\left(\mathbf{z'_{u,i}}\bm\delta\right) \\
\sigma_{v,i}^2 &=\exp\left( \mathbf{z'_{v, i}}\bm\varphi\right)
\end{align*}
%
Allowing heteroscedasticity in the variance of the one-sided error term $u$ 
allows to examine the determinants of inefficiency. This parametrization has 
been discussed in \citet{reif91, cau93, cau95, had99}. Moreover, according to 
\citet{wang02}, not accounting for the heteroscedasticity in the efficiency  
component can lead to inconsistent results. As underlined in 
\citet[p.~115]{kum14}, "the early literature adopts a two-step procedure to investigate 
the relationship", which was later proved by \citet{wang02} to poduce biased results.
In the case of the truncated normal distribution, \citet{kum91, hua94, batt95}
suggested to only parameterize the mean of the distribution as 
$\mu_i = \mathbf{z}'_{\bm\mu, \mathbf{i}}\bm\zeta$ when examining the 
inefficiency determinants. However, to account for non-monotonicity in the 
determinants of inefficiency, \citet{wang02b} suggested to parameterized both the 
mean and and the variance of the pre-truncated distribution. The same process
can also be applied to the log-normal distribution.

Regarding the truncated normal distribution, \citet{wang02} have formulated a 
different strategy for modelling the inefficiency variable:

$$u_i \sim h(\mathbf{z}_i, \bm\delta)u_*$$

where $h() \geq 0$ (\textit{scaling factor}) is observation-specific and 
non-stochastic and $u_*$ is a random (\textit{common}) variable that follows 
here a truncated normal distribution $\mathcal{N}^+\left(\tau, \sigma_u^2\right)$. 
Following \cite{kum14}, we re-parameterize the truncated normal distribution as 
$\mathcal{N}^+\left(\tau, \exp{\left(c_u\right)}\right)$. Then

$$u \equiv \exp\left(\mathbf{z'_{u,i}}\bm\delta\right)\mathcal{N}^+
\left(\tau, \exp{\left(c_u\right)}\right)=\mathcal{N}^+\left(\tau\exp\left(
\mathbf{z'_{u,i}}\bm\delta\right), \exp{\left(c_u + 2\mathbf{z'_{u,i}}
\bm\delta\right)}\right)$$

The advantage of the scaling property is that $u_*$, also known as the \textit{base
inefficiency}, anhd its distribution the \textit{base distribution}
\citep{wang02, al06}, does not depend on $\mathbf{z}$. This implies
that the distribution of $u_i$ has the same shape for all the firms, and the 
scaling function $h(\dot)$ expands or contracts the scale of the 
distribution.\footnote{According to \citet[p.~203]{al06}, "the basic random 
variable $u_*$ can be seen as the firms's base efficiency level which captures 
things like the manager's natural skills, which we view as random. How well 
these natural skills are exploited to manage the firm efficiently depends on the 
other variables $\mathbf{z}$, which might include the manager's education or 
experience, or measures of the environment in which the firm operates, for example."} 
When determinants of inefficiency are evaluated, so does the marginal impact of 
each $z$ variable on inefficiency. This marginal impact is estimated using the
derivatives $\partial E\left[u_i\right]/\partial z\left[k\right]$ and
$\partial V\left[u_i\right]/\partial z\left[k\right]$, which vary 
depending the distribution of $u$.



\subsubsection{Efficiency estimation} 

After estimating the model parameters, the other interest is to derive observation 
specific (in)efficiency. If we take the example of the half-normal distribution, 
after the estimation, we can obtain a value for $\sigma_u^2$, which is enough if
one is interested in the average inefficiency 
($E\left[u\right]=\sigma_u\sqrt{\frac{2}{\pi}}$). However, to obtain individual
observation inefficiency, $\sigma_u^2$ is not anymore enough.\footnote{Unless 
inefficiency determinants are considered.} An early solution was proposed by \citet{jon82}
who suggested to compute $E\left[u_i|\hat{\epsilon}_i\right]$. \citet{jon82} also 
formulated an alternative to the conditional mean estimator, viz., the 
conditional mode $M\left[u_i|\hat{\epsilon}_i\right]$. Once point estimates of 
$u_i$ are obtained, technical efficiency can be computed as 
$\exp{\left(-\hat{u}_i\right)}$. Later, \citet{batt88} proposed to 
derive the conditional mean efficiency using $E\left[\exp{\left(-u_i\right)|\hat{\epsilon}_i}\right]$.
We went a step further in the \pkg{sfaR} package to also provide the reciprocal
mean efficiency: $E\left[\exp{\left(u_i\right)|\hat{\epsilon}_i}\right]$.
Also whenever possible we also provide the confidence intervall of the (in)efficiency
scores \citep{horrace96}. Table~\ref{table:effcross}, in the Appendix, summarizes the main (in)efficiency measures. For the 
weibull and log-normal distributions, all (in)efficiency measures are evaluated
by solving a numerical integration (in \pkg{sfaR}, this 
is solved using adaptive multidimensional integration over hypercubes). For the
uniform distribution, different (in)efficiency scores can be derived when
$\theta \rightarrow \infty$.


\subsection{Panel stochastic frontier models}

If all the previous model are also valid for pooled cross-sectional data, they 
do not take advantage of the panel structure of the data when this is available. 
In the case of the panel data, some of the latent unobserved heterogeneity can 
be accounted for as inefficiency or individual specific heterogeneity. In the 
\pkg{sfaR}, we group the panel models into three categories. The first generation
of panel models consider the time invariant heterogeneity as if it were inefficiency, 
the second generation of models simultaneously consider firm heterogeneity in addition 
to inefficiency. Finally, the third generation model considers firm heterogeneity, 
and persistent and transient inefficiency components. All the panel models 
considered in \pkg{sfaR} are distribution-based models.

\subsubsection{Time-invariant inefficiency models}

The first to consider the panel dimension when estimating SF is \citet{pl81} who
considered the following model

\begin{equation} \label{eq:2}
y_{it} = \mathbf{x_{it}'}\bm{\beta} + v_{it} - Su_i \quad \text{with} \quad i = 1, \cdots, N \quad \text{and} \quad t = 1, \cdots, T
\end{equation}

where the random effect $u_i$ is considered to be the inefficiency. If this model
has only been developed for a few distributions for $u_i$, e.g., half-normal \citep{pl81}, 
truncated normal \citep{batt88}, rayleigh \citep{hajar15}, uniform \citep{nguyen10}, we extend it in the \pkg{sfaR} package to all the
other distributions.\footnote{Proofs for the derivation of the density function
of $f(\epsilon)$ is available from the authors upon request.} 
An overview of the density of $\epsilon_i$ can be found in 
Table~\ref{table:mlpl81}, in the Appendix.\footnote{A distribution-free 
estimation of this model has been discussed in \citet{sl84}.}

Heterogeneity and heteroscedasticity can be handled in the \citet{pl81} model.
However, the determinant variables $\mathbf{z}$ have to be time invariant \citep{kum2000}. In the
\pkg{sfaR}, we suggest several ways to obtain time invariant determinants (see section \ref{sec:prespackage}).
Regarding the (in)efficiency scores, they are computed the same way as in the 
cross-sectional models using the formulas in Table~\ref{table:effcross} in the 
Appendix.

\subsubsection{Time-varying inefficiency models}

Assuming that inefficiency is persistent over time may be quite restrictive 
especially for moderate to large time span data. It also implies that DMUs
never learn over time. The time-varying inefficiency models overcome the 
previous issues, and can be represented by equation~\ref{eq:3}.

\begin{equation} \label{eq:3}
y_{it} = \mathbf{x_{it}'}\bm{\beta} + v_{it} - SG(t)u_i \quad \text{with} \quad i = 1, \cdots, N \quad \text{and} \quad t = 1, \cdots, T
\end{equation}

All the developments for time-varying inefficiency models are based on the 
functional form assumed for $G(t)$.\footnote{A distribution-free approach has 
been discussed in \citet{css90, lees93}.} The different possibilities of $G(t)$ in 
the \pkg{sfaR} are summarized in Table~\ref{table:gt}. \citet{kum90} was the 
first to suggest an ML estimation of time-varying inefficiency. Later \citet{batco92}
developed the "time-decay" model along with two other alternatives. Many other
models were suggested as documented in Table~\ref{table:gt}. 

\begin{table}
%\setlength{\arrayrulewidth}{.01em}
\renewcommand{\arraystretch}{1.3}
\centering
\begin{tabular}{@{}cc@{}}
\toprule
References & $G(t)$ \\[1em]
\midrule
\citet{kum90} & $G(t)=\left[1+\exp{\left(\eta_1t + \eta_2t^2\right)}\right]^{-1}$ \\[1em]
%\hdashline
\citet{batco92} "time-decay" & $G(t)=\exp{\left[-\eta(t-T)\right]}$ \\[1em]
%\hdashline
\citet{batco92} & $G(t)= 1 + \eta_1(t-T) + \eta_2(t-T)^2$ \\[1em]
%\hdashline
Modified \citet{lees93} & $G(t)=\exp{\left[-\eta_t(t-T)\right]}$ \\[1em] 
%\hdashline
\citet{cue00} & $G(t)=\exp{\left[-\eta_i(t-T)\right]}$ \\[1em] 
%\hdashline
\citet{cue02, feng09} & $G(t)=\exp{\left[-\eta_1(t-T)-\eta_2(t-T)^2\right]}$ \\[1em] 
%\hdashline
\citet{kw05} & $G(t)=\exp{\left[-\eta(t-t_1)\right]}$ \\[1em]
%\hdashline 
\citet{al06} & $G(t)=\exp{\left(\bm{\eta}'\mathbf{z}_{git}\right)}$ \\[1em]
\bottomrule
\end{tabular}
\caption{List of functional forms for $G(t)$}
\label{table:gt}
\end{table}

In \pkg{sfaR}, we also presented a modified version of the \citet{lees93} model. 
For this model, the last period parameter is not identifiable so we set 
$\eta_T=1$ \citep[p.~304]{par14}. As previously, \pkg{sfaR} includes
time-varying inefficiency for all the ten distributions initially mentioned.
In terms of efficiency determinants, the only model that allows for time-varying
efficiency driver is the one by \citet{al06}. For all the other models, efficiency
drivers are modelled as in the case of the time-invariant inefficiency models. 
For the (in)efficiency estimates, see formula in Table~\ref{table:effpanel}.

With the time-invariant inefficiency models, the ones developed in this subsection
constitute the first generation of panel SF models in the \pkg{sfaR} package.

%\subsection{Second generation of panel SF models}

%\subsection{Third generation of panel SF models}

\section{Latent class stochastic frontiers and variants} \label{sec:modelcm}

\subsection{Latent class stochastic frontier}

The latent class SF model (LCM) handles production heterogeneity
through a mixture of production functions. The standard finite mixture models have
been extended to the SF framework by \citet{cau03, orea04}. In the next subsections, 
we describe the LCM for both pooled cross-section and panel data.

\subsubsection{For cross-sectional data}

In the case of a pooled cross-section data, the equation associated with the LCM can
be written as

\begin{equation} \label{eq:4}
y_{it} = \mathbf{x_{it}'}\bm{\beta}_j + v_{it,j} - Su_{it,j} \quad \text{with} \quad i = 1, \cdots, N \quad \text{and} \quad j = 1, \cdots, J
\end{equation}

In Equation~\ref{eq:4}, the LCM is implied by a $j$ class-specific
technological parameters. Assuming the half-normal distribution for the 
inefficiency term, the contribution of observation $i$ in period $t$ to the 
likelihood conditional on class $j$ is defined as: 

\begin{equation}\label{eq:5}
P(it|j) = \frac{2}{\sqrt{\sigma_{u|j}^2 + 
 \sigma_{v|j}^2}}\phi\left(\frac{S\epsilon_{it|j}}{\sqrt{
 \sigma_{u|j}^2 +\sigma_{v|j}^2}}\right)\Phi\left(\frac{
 \mu_{it*|j}}{\sigma_{*|j}}\right)
\end{equation}

where 

$$\mu_{it*|j}=\frac{- S\epsilon_{it|j}\sigma_{u|j}^2}{\sigma_{u|j}^2 +
\sigma_{v|j}^2}$$

 $$\sigma_*^2 = \frac{\sigma_{u|j}^2 \sigma_{v|j}^2}{\sigma_{u|j}^2 + 
 \sigma_{v|j}^2}$$

The distribution of observations into classes is based
on prior probability that can depend on some covariates. This probability can 
be modelled using a logit distribution as

\begin{equation} \label{eq:6}
\pi_{itj} = \frac{\exp{\left(\mathbf{q}'_{it}\bm{\lambda}_j\right)}}{
\sum_{m=1}^J \exp{\left(\mathbf{q}'_{it}\bm{\lambda}_m\right)}
}
\end{equation}

such that $\bm{\lambda}_J = 0$. 

The unconditional likelihood of observation $i$ in period $t$ is simply the average
 over the $J$ classes:

 $$P(it) = \sum_{m=1}^{J}\pi(it,m)P(it|m)$$

 The number of classes to retain can be based on information criterion (AIC, BIC).
 Moreover, class assignment is based on the largest posterior probability. This
 probability is obtained using Bayes' rule, as follows for class $j$:

 $$w\left(j|it\right)=\frac{P\left(it|j\right)
 \pi\left(it,j\right)}{\sum_{m=1}^JP\left(it|m\right)
 \pi\left(it, m\right)}$$
 
 As in the standard cases, heteroscedasticity can be accommodated during the 
 estimation. It is worth noting that the \pkg{sfaR} package, currently, only considers
 the latent class SF model with the half-normal distribution.

\subsubsection{For panel data}

The difference with the pooled LCM is that the probability of belonging to 
a class is fixed over time. The logit distribution is \ref{eq:6} becomes

\begin{equation} \label{eq:7}
\pi_{ij} = \frac{\exp{\left(\mathbf{q}'_{i}\bm{\lambda}_j\right)}}{
\sum_{m=1}^J \exp{\left(\mathbf{q}'_{i}\bm{\lambda}_m\right)}
}
\end{equation}

In the \pkg{sfaR} package, the time invariant covariates are obtained by taking
the average of the variables for each cross-section. As previously, only the 
half-normal distribution is implemented. The latent class SF models in the 
case of panel data is centered around the first generation panel models. For
example, in the case of time-varying inefficiency models, the the contribution 
of cross section $i$ to the likelihood conditional on class $j$ is defined as: 

\begin{equation}\label{eq:8}
 P(i|j) = \frac{2\sigma_*}{\sigma_u\sigma_v^T\left(2\pi\right)^{\frac{T}{2}}}\exp{\left[-\frac{1}{2}\left(-\frac{\mu_{i*}^2}{\sigma_*^2} + \frac{\sum_{t=1}^T\epsilon_{it}^2}{\sigma_v^2}\right)\right]}\Phi\left(\frac{\mu_{i*}}{\sigma_*}\right)
\end{equation}

where

$$\mu_{i*} = \frac{-S\sigma_u^2\sum_{t=1}^TG(t)\epsilon_{it}}{\sigma_u^2\sum_{t=1}^TG(t)^2 + \sigma_v^2}$$

and

$$\sigma_*^2=\frac{\sigma_u^2\sigma_v^2}{\sigma_u^2\sum_{t=1}^TG(t)^2 + \sigma_v^2}$$

As previously, the unconditional likelihood of cross-section $i$ is simply the average
 over the $J$ classes:

$$P(i) = \sum_{m=1}^{J}\pi(i,m)P(i|m)$$

The panel LCM model also handles heteroscedasticity in the error terms.

\subsection{Zero inefficiency stochastic frontier}

The zero inefficiency SF model (ZISF) has been introduced by \citet{kpt13}. The 
philosophy of the model is that there are two groups of observations, one that is
efficient and the other is not. The probability of being in one of the group can 
depend, as in the LCM case, on some covariates. Technically we have

\begin{equation}\label{eq:9}
y_i = \begin{cases}
\alpha + \mathbf{x_i^{\prime}}\bm{\beta} + 
 v_i - Su_i & \text{with probability} \quad p \\
 \alpha + \mathbf{x_i^{\prime}}\bm{\beta} +
  v_i & \text{with probability} \quad 1-p
\end{cases}
\end{equation}

The prior probability of belonging to the inefficient class can depend on 
some covariates using four possible specifications: 

\begin{itemize} \itemsep 12pt
\item logit specification

$$p_i = \frac{\exp{(\mathbf{z}_{i}^{\prime}\bm{\theta})}}{1-
\exp{(\mathbf{z}_{i}^{\prime}\bm{\theta})}}$$

\item probit specification

$$p_i = \Phi\left(\mathbf{z}_{i}^{\prime}\bm{\theta}\right)$$

\item cauchit specification

$$p_i = 1/\pi\arctan(\mathbf{z}_{i}^{\prime}\bm{\theta})+1/2$$

\item cloglog specification

$$p_i = 1-\exp\left(-\exp(\mathbf{z}_{i}^{\prime} \bm{\theta})\right)$$

\end{itemize}

In the case of the truncated normal distribution, the convolution of $u_i$ and 
$v_i$ is:

 $$f(\epsilon_i)=\frac{p_i}{\sqrt{\sigma_u^2 + 
 \sigma_v^2}}\phi\left(\frac{S\epsilon_i + \mu}{\sqrt{
 \sigma_u^2 + \sigma_v^2}}\right)\Phi\left(\frac{
 \mu_{i\ast}}{\sigma_\ast}\right)\Big/\Phi\left(
 \frac{\mu}{\sigma_u}\right) + \frac{1-p_i}{\sigma_v}
 \Phi\left(\frac{S\epsilon}{\sigma_v}\right)$$

 where

 $$\mu_{i*}=\frac{\mu\sigma_v^2 - 
 S\epsilon_i\sigma_u^2}{\sigma_u^2 + \sigma_v^2}$$

 and

 $$\sigma_*^2 = \frac{\sigma_u^2 
 \sigma_v^2}{\sigma_u^2 + \sigma_v^2}$$
 
 The ZISF is also implemented for the other nine distributions in the package.
 As in the case of LCM, class assignment is based on the largest posterior 
 probability. The model presented so far assumed common noise variable $v$ for 
 the two classes of observation. \pkg{sfaR} allows to estimate the ZISF model
 with different noise term depending on the class.\footnote{For now the
 \pkg{sfaR} package estimates the ZISF for cross-section or pooled data.}

\subsection{Contaminated noise stochastic frontier}

The contaminated noise SF (CNSF), is another variant for the LCM first discussed 
by \citet{wheat19} to handle outliers. Here the observations are split into 
two groups for which the idiosyncratic error is different. Practically we have

\begin{equation}\label{eq:10}
y_i = \begin{cases}
\alpha + \mathbf{x_i^{\prime}}\bm{\beta} + 
 v_{1i} - Su_i & \text{with probability} \quad p \\
 \alpha + \mathbf{x_i^{\prime}}\bm{\beta} + 
v_{2i} - Su_i & \text{with probability} \quad 1-p
\end{cases}
\end{equation}

All the other developments are similar to the ZISF case. The prior probability
of belonging to a class can be specified using either the logit, probit, cauchit
and cloglog distributions. The CNSF is also available for the ten distributions.

A slight extension of the CNSF model compared to \citet{wheat19}, is that the 
\pkg{sfaR} allows for the possibility to also have different inefficiency term 
between the two groups of observations.

\subsection{Multi-modal inefficiency stochastic frontier}

The multi-modal inefficiency SF (MISF) is an extension in the \pkg{sfaR} package, 
which splits the observations into two groups, but with different inefficiency
error term. Basically we have:

\begin{equation}\label{eq:11}
y_i = \begin{cases}
\alpha + \mathbf{x_i^{\prime}}\bm{\beta} + 
 v_{i} - Su_{1i} & \text{with probability} \quad p \\
 \alpha + \mathbf{x_i^{\prime}}\bm{\beta} + 
v_{i} - Su_{2i} & \text{with probability} \quad 1-p
\end{cases}
\end{equation}

All the other derivations are similar to the ZISF and CNSF previously introduced.

\section{Other models} \label{sec:other}

\subsection{Metafrontiers}

\subsection{Sample selection stochastic frontier} \label{subsec:ss}


\section[Introduction to the sfaR package]{Introduction to the \pkg{sfaR} package} \label{sec:prespackage}

\subsection{Interface to main functions}

The \pkg{sfaR} package embeds ten main functions for the different groups of SFA
models. The name of each function is chosen so that its purpose is very 
explicit. For cross-sectional or pooled data we have the following functions:
%
\begin{itemize} \itemsep 10pt
\item \code{sfacross} for the standard SFA models 
\item \code{sfalcmcross} for latent class SF models
\item \code{sfagzisfcross} for the generalized zero-inefficiency SF models
\item \code{sfacnsfcross} for the contaminated noise SF models
\item \code{sfamisfcross} for the multi-modal inefficiency SF models
\item \code{sfazisfcross} for the zero-inefficiency SF models
\item \code{sfametacross} for the metafrontier models
\item \code{sfaselectioncross} for the sample selection SF model.
\end{itemize}
%
In the case of the panel data, two/three main functions are implemented in \pkg{sfaR} 
%
\begin{itemize} \itemsep 10pt
\item \code{sfapanel1} for the first generation SF models using panel data
\item \code{sfalcmpanel} for the latent class SF models using panel data.
\end{itemize}
%
All these functions are formula based for the estimation of the parameters of the 
production technology. Moreover, linearity is assumed for all the parameters.
In most cases \code{formula} is used to define the main part of the technology. 
\code{sfaselectioncross} requires \code{selectionF} for the selection
equation, and \code{frontierF} for the frontier equation. Heteroscedasticity in 
the variances of the one-sided and the two-sided error terms $u$ and $v$ are 
also defined using one-sided formulas (only right arguments e.g. \code{~x}) 
with \code{uhet} and \code{vhet}, respectively. Similarly, when some 
distributions are chosen, heterogeneity in the inefficiency term can be 
introduced using the one-sided formula based option \code{muhet}. By default, 
it is assumed that a log-linear function is to be estimated (e.g. Cobb-Douglas, 
or Translog). If this is not the case, the user can set the argument 
\code{logDepVar} to \code{FALSE} (e.g. quadratic production function).

For consistency with the common \code{lm} functions, arguments \code{data} (which 
must be a \code{data.frame}), \code{subset}, and \code{weights} are also 
present. Several additional arguments, some common between the functions and 
other specific, allow many possibilities in the model to estimate and how the 
estimation can be conducted. For instance \code{udist} allows the choice of 
the distribution for the one-sided error term $u$. For functions like 
\code{sfacross}, \code{sfacnsfcross}, \code{sfamisfcross}, \code{sfazisfcross}, 
and \code{sfametacross}, the ten aforementioned distributions are possible:
%
\begin{itemize} \itemsep 10pt
\item \code{"hnormal"} for the half-normal distribution
\item \code{"exponential"} for the exponential distribution
\item \code{"tnormal"} for the truncated normal distribution
\item \code{"rayleigh"} for the rayleigh distribution
\item \code{"uniform"} for the uniform distribution
\item \code{"gamma"} for the gamma distribution
\item \code{"lognormal"} for the lognormal distribution
\item \code{"weibull"} for the weibull distribution
\item \code{"genexponential"} for the generalized exponential distribution
\item \code{"tslaplace"} for the truncated skewed-laplace distribution.
\end{itemize}
%
For \code{sfacross}, the scaling property discussed by \cite{wang02} is obtained
by setting the \code{scaling} to \code{TRUE}. In the case of the gamma, 
log-normal, and weibull distributions, the log-likelihood is estimated using 
simulation. By default, the quasi-random numbers are draws from 
Halton sequences (\code{simType = "halton"}). Seven other possibilities can be 
used:
%
\begin{itemize} \itemsep 10pt
\item \code{"ghalton"} for generalized Halton sequences
\item \code{"sobol"} for Sobol sequences
\item \code{"rsobol"} for randomized Sobol sequences \footnote{The generalized
Halton and the Sobol sequences are derived from the \pkg{qrng} package \citep{qrng}.}
\item \code{"richtmyer"} for Richtmyer sequences
\item \code{"rrichtmyer"} for randomized Richtmyer sequences \footnote{Richtmyer sequences
are obtained from the \pkg{mnorm} package \citep{mnorm}.}
\item \code{"uniform"} for uniform draws 
\item \code{"mlhs"} for the Modified Latin Hypercube sequences.
\end{itemize}
%
For each of the random number generator, the number of draws can be chosen with
the argument \code{Nsim} (\code{100} by default).\footnote{For more 
possibilities on the draws see corresponding function help in the package
documentation.}

Regarding the \code{sfalcmcross}, \code{sfagzisfcross}, and the 
\code{sfaselectioncross}, due to their complexity, only the half-normal 
distribution is currently possible. For the case of \code{sfalcmcross}, and 
\code{sfagzisfcross}, up to five classes can be estimated (\code{lcmClasses}, 
and \code{gzisfClasses} arguments). For all the other variants of the latent
classes SF (CNSF, MISF, and ZISF), as presented in the previous section, two 
classes of observations are estimated. While for the \code{sfalcmcross} and 
\code{sfagzisfcross} functions, the probability of belonging to a class is 
parameterized as a logit model, for \code{sfacnsfcross}, \code{sfamisfcross}, and
\code{sfazisfcross}, four models are possible through the \code{linkF} argument:
%
\begin{itemize} \itemsep 10pt
\item "logit" for the logit model (default)
\item "probit" for the probit model
\item "cauchit" for the cauchit model
\item "cloglog" for the cloglog model
\end{itemize}
%
In the case of the \code{sfacnsfcross} function, the \code{sigmauType} argument
allows to have \code{"common"} or \code{"different"} parameters for the 
inefficiency distribution among the two classes. For the \code{sfazisfcross}, 
the argument \code{sigmavType} allows the same thing but for the two-sided error
term $v$.

\code{sfaselectioncross}

\code{sfametacross}

As mentioned in the Introduction section, an advantage of the \pkg{sfaR} is to
offer multiple possibilities for the optimization algorithm used to solve the
M(S)L. Currently, eleven algorithms are included with the argument \code{method}
%
\begin{itemize} \itemsep 10pt
\item \code{"bfgs"} for Broyden-Fletcher-Goldfarb-Shanno (default algorithm)
\item \code{"bhhh"} for Berndt-Hall-Hall-Hausman
\item \code{"nr"} for Newton-Raphson
\item \code{"nm"} for Nelder-Mead
\item \code{"cg"} for Conjugate Gradient
\item \code{"sann"} for Simulated Annealing\footnote{"bfgs", "bhhh", "nr", "nm", 
"cg", and "sann" algorithms are derived from the \pkg{maxLik} package \citep{maxlik}.} 
\item \code{"ucminf"} for a quasi-Newton type optimization with BFGS updating 
of the inverse Hessian and soft line search with a trust region type monitoring 
of the input to the line search algorithm \footnote{"ucminf" is implemented in 
the \pkg{ucminf} package \citep{ucminf}.}
\item \code{"mla"} for general-purpose optimization based on Marquardt-Levenberg 
algorithm \footnote{"mla" is implemented in 
the \pkg{marqLevAlg} package \citep{mla}.}
\item \code{"sr1"} for Symmetric Rank 1 
\item \code{"sparse"} for trust regions and sparse Hessian \footnote{"sr1" and 
"sparse" are implemented in the \pkg{trustOptim} package \citep{trust}.}
\item \code{"nlminb"} for optimization using PORT routines. \footnote{"nlminb" 
is implemented in the \pkg{stats} package \citep{stats}.}
\end{itemize}
%

As non-linear optimization is well known to converge at local optimum, the
\pkg{sfaR} allows the user to set the starting values\footnote{In most cases the 
starting values are obtained from the method of moments or from a simpler M(S)L 
estimation}, and to allow for a random start (\code{randStart} argument). The
random starting values are obtained by taking the initial starting values and 
adding random numbers drawn from a normal distribution with zero mean and 
standard value of $0.01$.

The \code{sfacross} encloses all the previous distributions and provide many 
additional options for the M(S)L estimation. It is a formula based function.
The help file of the function is a good starting point to have an overview of 
all the possibilities. For the illustration, we first consider the utility data 
contained in the package, and replicate Table 4.1 in . 
First load the package using 

\subsection{Inherited methods}

Two main methods are directly associated with SF estimation functions. The first
method is \code{efficiencies} to compute several efficiency variables. In all 
the cases the inefficiency scores are derived following \cite{jon82}. Using
the latter, the efficiency scores are obtained using 
$\exp{E\left[\left(-u|\epsilon\right)\right]}$. Another conditional efficiency
scores variable is also derived following \cite{batt95}. 
reciprocal efficiencies.

in some cases and depending on the distribution, the mode and the confidence
intervals associated with the (in)efficiency scores are also returned.

In the case of the LCM and its variants, additional variables are returned: 
the posterior probability of beloging to any class, which is then used to affect 
every observation to the class with the highest probability.

\section{Applications} \label{sec:illus}

\subsection[Replication of Kumbhakar et al.2014, p. 119]{Replication of~\citet[p.~119]{kum14}}

For the first illustration of the capacities of the \pkg{sfaR}, we replicate the 
the results in Table 4.1 in \cite[p.~119]{kum14}. The data used is on fossil fuel
fired steam electric power generation plants in the United States. The dataset
is part of the package under the name \code{utility}, and is well described under
the help file associated with it.\footnote{We thank \cite{kum14} for allowing
us to use the data in the package.} The estimated model is a cost function 
considering one output and three input prices (labor and maintenance, fuel, 
and capital). A translog cost function is estimated with a normalization using
fuel input price to ensure the price homogeneity property of cost functions. For
the SF models, three distributions are considered: half-normal, exponential, and
truncated normal. For the latter distribution, the scaling property is also 
considered. Hence, all the models are estimated by also estimating the effect of
regulation (dummy variable \code{regu}) on the inefficiency. The code to estimate
the four models, in addition to the OLS one are presented below:

\begin{Schunk}
\begin{Sinput}
R> ols <- lm(log(tc/wf) ~ log(y) + I(1/2 * (log(y))^2) +
+   log(wl/wf) + log(wk/wf) + I(1/2 * (log(wl/wf))^2) + I(1/2 * (log(wk/wf))^2) +
+   I(log(wl/wf) * log(wk/wf)) + I(log(y) * log(wl/wf)) + I(log(y) * log(wk/wf)), 
+   data = utility)
R> hlf <- sfacross(formula = log(tc/wf) ~ log(y) + I(1/2 * (log(y))^2) +
+   log(wl/wf) + log(wk/wf) + I(1/2 * (log(wl/wf))^2) + I(1/2 * (log(wk/wf))^2) +
+   I(log(wl/wf) * log(wk/wf)) + I(log(y) * log(wl/wf)) + I(log(y) * log(wk/wf)),
+   udist = 'hnormal', uhet = ~ regu, data = utility, S = -1, method = 'bfgs')
R> trnorm <- sfacross(formula = log(tc/wf) ~ log(y) + I(1/2 * (log(y))^2) +
+   log(wl/wf) + log(wk/wf) + I(1/2 * (log(wl/wf))^2) + I(1/2 * (log(wk/wf))^2) +
+   I(log(wl/wf) * log(wk/wf)) + I(log(y) * log(wl/wf)) + I(log(y) * log(wk/wf)),
+   udist = 'tnormal', muhet = ~ regu, data = utility, S = -1, method = 'bfgs')
R> tscal <- sfacross(formula = log(tc/wf) ~ log(y) + I(1/2 * (log(y))^2) +
+   log(wl/wf) + log(wk/wf) + I(1/2 * (log(wl/wf))^2) + I(1/2 * (log(wk/wf))^2) +
+   I(log(wl/wf) * log(wk/wf)) + I(log(y) * log(wl/wf)) + I(log(y) * log(wk/wf)),
+   udist = 'tnormal', muhet = ~ regu, uhet = ~ regu, data = utility, 
+   S = -1, method = 'bfgs', scaling = TRUE)
R> expo <- sfacross(formula = log(tc/wf) ~ log(y) + I(1/2 * (log(y))^2) +
+   log(wl/wf) + log(wk/wf) + I(1/2 * (log(wl/wf))^2) + I(1/2 * (log(wk/wf))^2) +
+   I(log(wl/wf) * log(wk/wf)) + I(log(y) * log(wl/wf)) + I(log(y) * log(wk/wf)),
+   udist = 'exponential', uhet = ~ regu, data = utility, S = -1, method = 'bfgs')
\end{Sinput}
\end{Schunk}

Before presenting the results of the previously estimated model, let's have a 
glance at the OLS residuals. Basically here, since we are estimating a cost 
function we expect the OLS residuals to have the "right" positive skeweness. 
A plot of the OLS residuals, and for a symmetric normal distribution can be 
found on Figure~\ref{fig:1}. In this particular case, we are "lucky" that the 
OLS residuals exhibit the "right" positive skewness.

\begin{figure}[t!]
\centering
\includegraphics{sfaR_handout-densityplots}
\caption{\label{fig:1} OLS residuals plot vs a symmetric normal distribution.}
\end{figure}

The estimation of the four different SF models (half-normal, truncated normal, 
truncated normal with scaling property, and exponential) is conducted using the 
default \textit{bfgs} optimization algorithm. The \code{summary} method returns
and prints in light of the \code{lm} function, the summary of estimation. 
Additionally, more information to assess the quality of the estimation is 
provided, e.g. the gradient norm, or the condition number associated with the 
hessian matrix. In the case of the object obtained using the half normal 
distibution the summary is

\begin{Schunk}
\begin{Sinput}
R> summary(hlf)
\end{Sinput}
\begin{Soutput}
-------------------------------------------------------------------------------- 
Normal-Half Normal SF Model 
Dependent Variable:                                                   log(tc/wf) 
Log likelihood solver:                                         BFGS maximization 
Log likelihood iter:                                                         169 
Log likelihood value:                                                   73.86328 
Log likelihood gradient norm:                                        3.79396e-03 
Hessian condition number:                                            1.60662e+08 
Estimation based on:                                        N =  791 and K =  13 
Inf. Cr:                                         AIC  =  -121.7 AIC/N  =  -0.154 
                                                  BIC  =  -61.0 BIC/N  =  -0.077 
                                                  HQIC =  -98.4 HQIC/N =  -0.124 
-------------------------------------------------------------------------------- 
Variances: Sigma-squared(v)   =                                          0.00544 
           Sigma(v)           =                                          0.00544 
           Sigma-squared(u)   =                                          0.15807 
           Sigma(u)           =                                          0.15807 
Sigma = Sqrt[(s^2(u)+s^2(v))] =                                          0.40435 
Gamma = sigma(u)^2/sigma^2    =                                          0.96676 
Lambda = sigma(u)/sigma(v)    =                                          5.39278 
Var[u]/{Var[u]+Var[v]}        =                                          0.91355 
Variances averaged over observations 
-------------------------------------------------------------------------------- 
Average inefficiency E[ui]     =                                         0.31722 
Average efficiency E[exp(-ui)] =                                         0.74777 
-------------------------------------------------------------------------------- 
Stochastic Cost Frontier, e = v + u 
-----[ Tests vs. No Inefficiency ]-----
Likelihood Ratio Test of Inefficiency
Deg. freedom for inefficiency model                                            2 
Log Likelihood for OLS Log(H0) =                                       -43.88803 
LR statistic:  
Chisq = 2*[LogL(H0)-LogL(H1)]  =                                       235.50261 
Kodde-Palm C*:       95%: 5.13838                                   99%: 8.27327 
Coelli (1995) skewness test on OLS residuals
M3T: z                         =                                        10.41708 
M3T: p.value                   =                                         0.00000 
Final maximum likelihood estimates 
-------------------------------------------------------------------------------- 
                         Deterministic Component of SFA 
-------------------------------------------------------------------------------- 
                           Coefficient Std. Error z value Pr(>|z|)   
(Intercept)                    4.08258    3.55331  1.1489 0.250575   
log(y)                         0.31764    0.23957  1.3259 0.184883   
I(1/2 * (log(y))^2)            0.02403    0.01331  1.8056 0.070987 . 
log(wl/wf)                     0.68589    0.89834  0.7635 0.445158   
log(wk/wf)                     2.19635    1.22233  1.7969 0.072359 . 
I(1/2 * (log(wl/wf))^2)       -0.05335    0.18573 -0.2873 0.773916   
I(1/2 * (log(wk/wf))^2)        0.61425    0.27723  2.2156 0.026717 * 
I(log(wl/wf) * log(wk/wf))     0.49385    0.17811  2.7727 0.005559 **
I(log(y) * log(wl/wf))         0.04818    0.03018  1.5964 0.110388   
I(log(y) * log(wk/wf))        -0.06686    0.04171 -1.6032 0.108898   
-------------------------------------------------------------------------------- 
                  Parameter in variance of u (one-sided error) 
-------------------------------------------------------------------------------- 
                           Coefficient Std. Error  z value  Pr(>|z|)
Zu_(Intercept)                -2.61708    0.11081 -23.6179 < 2.2e-16
Zu_regu                        1.05260    0.11702   8.9948 < 2.2e-16
                              
Zu_(Intercept)             ***
Zu_regu                    ***
-------------------------------------------------------------------------------- 
                 Parameters in variance of v (two-sided error) 
-------------------------------------------------------------------------------- 
                           Coefficient Std. Error z value  Pr(>|z|)
Zv_(Intercept)                -5.21487    0.21616 -24.125 < 2.2e-16
                              
Zv_(Intercept)             ***
---
Signif. codes:  0 '***' 0.001 '**' 0.01 '*' 0.05 '.' 0.1 ' ' 1
-------------------------------------------------------------------------------- 
Model was estimated on : août sam. 12, 2023 at 16:51:24 
Log likelihood status: successful convergence  
-------------------------------------------------------------------------------- 
\end{Soutput}
\end{Schunk}

The \pkg{sfaR} includes an \code{extract} option compatible with the 
\pkg{texreg} package \citep{texreg13}. The following code generates Table~\ref{table:3}.

\begin{Schunk}
\begin{Sinput}
R> library(texreg)
R> texreg(list(ols, hlf, trnorm, tscal, expo), stars = c(0.01, 0.05, 0.1), digits = 3, 
+    custom.model.names = c("ols", "hnormal", "tnormal", "trunc. scal.", "expo"), 
+    table = FALSE)
\end{Sinput}
\end{Schunk}

\begin{table}
\centering
\begin{adjustbox}{width=1\textwidth}
\small
%\begin{longtable}{l c c c c c}
\begin{tabular}{l c c c c c}
\hline
 & ols & hnormal & tnormal & trunc. scal. & expo \\
\hline
(Intercept)             & $8.467$       & $4.083$        & $4.511$        & $3.326$        & $2.866$        \\
                        & $(5.196)$     & $(3.553)$      & $(3.460)$      & $(3.496)$      & $(3.493)$      \\
log(y)                  & $-0.184$      & $0.318$        & $0.322$        & $0.376$        & $0.407^{*}$    \\
                        & $(0.336)$     & $(0.240)$      & $(0.233)$      & $(0.236)$      & $(0.236)$      \\
1/2 * (log(y))$^2$      & $0.066^{***}$ & $0.024^{*}$    & $0.019$        & $0.018$        & $0.015$        \\
                        & $(0.017)$     & $(0.013)$      & $(0.013)$      & $(0.013)$      & $(0.013)$      \\
log(wl/wf)              & $0.893$       & $0.686$        & $0.306$        & $0.558$        & $0.507$        \\
                        & $(1.326)$     & $(0.898)$      & $(0.855)$      & $(0.861)$      & $(0.849)$      \\
log(wk/wf)              & $2.374$       & $2.196^{*}$    & $2.127^{*}$    & $1.805$        & $1.573$        \\
                        & $(1.913)$     & $(1.222)$      & $(1.217)$      & $(1.229)$      & $(1.240)$      \\
1/2 * (log(wl/wf))$^2$  & $-0.087$      & $-0.053$       & $0.018$        & $-0.002$       & $0.024$        \\
                        & $(0.249)$     & $(0.186)$      & $(0.174)$      & $(0.176)$      & $(0.173)$      \\
1/2 * (log(wk/wf))$^2$  & $0.925^{**}$  & $0.614^{**}$   & $0.449$        & $0.451$        & $0.362$        \\
                        & $(0.464)$     & $(0.277)$      & $(0.284)$      & $(0.292)$      & $(0.299)$      \\
log(wl/wf) * log(wk/wf) & $0.359$       & $0.494^{***}$  & $0.484^{***}$  & $0.533^{***}$  & $0.553^{***}$  \\
                        & $(0.283)$     & $(0.178)$      & $(0.176)$      & $(0.179)$      & $(0.181)$      \\
log(y) * log(wl/wf)     & $0.028$       & $0.048$        & $0.058^{**}$   & $0.054^{*}$    & $0.057^{*}$    \\
                        & $(0.042)$     & $(0.030)$      & $(0.029)$      & $(0.030)$      & $(0.030)$      \\
log(y) * log(wk/wf)     & $-0.008$      & $-0.067$       & $-0.088^{**}$  & $-0.075^{*}$   & $-0.078^{*}$   \\
                        & $(0.061)$     & $(0.042)$      & $(0.042)$      & $(0.043)$      & $(0.044)$      \\
Zu\_(Intercept)         &               & $-2.617^{***}$ & $-1.254^{***}$ &                & $-3.775^{***}$ \\
                        &               & $(0.111)$      & $(0.303)$      &                & $(0.166)$      \\
Zu\_regu                &               & $1.053^{***}$  &                &                & $1.427^{***}$  \\
                        &               & $(0.117)$      &                &                & $(0.169)$      \\
Zv\_(Intercept)         &               & $-5.215^{***}$ & $-4.989^{***}$ & $-4.888^{***}$ & $-4.712^{***}$ \\
                        &               & $(0.216)$      & $(0.193)$      & $(0.200)$      & $(0.164)$      \\
Zmu\_(Intercept)        &               &                & $-1.318^{**}$  &                &                \\
                        &               &                & $(0.572)$      &                &                \\
Zmu\_regu               &               &                & $1.042^{***}$  &                &                \\
                        &               &                & $(0.322)$      &                &                \\
Zscale\_regu            &               &                &                & $0.625^{***}$  &                \\
                        &               &                &                & $(0.082)$      &                \\
tau                     &               &                &                & $-0.788$       &                \\
                        &               &                &                & $(0.646)$      &                \\
cu                      &               &                &                & $-1.658^{***}$ &                \\
                        &               &                &                & $(0.528)$      &                \\
\hline
R$^2$                   & $0.923$       & $$             & $$             & $$             & $$             \\
Adj. R$^2$              & $0.922$       & $$             & $$             & $$             & $$             \\
Num. obs.               & $791$         & $791$          & $791$          & $791$          & $791$          \\
AIC                     & $$            & $-121.727$     & $-142.304$     & $-131.367$     & $-129.840$     \\
BIC                     & $$            & $-60.974$      & $-76.877$      & $-65.941$      & $-69.087$      \\
log-likelihood          & $$            & $73.863$       & $85.152$       & $79.684$       & $77.920$       \\
\hline
\multicolumn{6}{l}{\scriptsize{$^{***}p<0.01$; $^{**}p<0.05$; $^{*}p<0.1$}}
\end{tabular}%\end{longtable}
\end{adjustbox}
\caption{Replication Table 4.1 in \citet[p.~119]{kum14}}
\label{table:3}
\end{table}

A summary of the (in)efficiency scores and the marginal impacts can be found in 
Table~\ref{table:4}.

\begin{table}[ht]
\centering
\scalebox{0.8}{
\begin{tabular}{lrrrrrr}
  \toprule
Distributions & $E\left[u|\epsilon\right]$ & $\exp{\left(-E\left[u|\epsilon\right]\right)}$ & $E\left[\exp{\left(-u\right)}|\epsilon\right]$ & $E\left[\exp{\left(u\right)}|\epsilon\right]$ & $\frac{\partial E[u]}{\partial regu}$ & $\frac{\partial V[u]}{\partial regu}$ \\ 
  \midrule
Half-Normal & 0.306 & 0.757 & 0.758 & 1.408 & 0.163 & 0.060 \\ 
  Truncated Normal & 0.278 & 0.777 & 0.779 & 1.368 & 0.208 & 0.065 \\ 
  Scaling & 0.268 & 0.784 & 0.786 & 1.355 & 0.167 & 0.077 \\ 
  exponential & 0.250 & 0.798 & 0.800 & 1.330 & 0.178 & 0.097 \\ 
   \bottomrule
\end{tabular}
}
\caption{Mean (in)efficiency scores and marginal impacts} 
\label{table:4}
\end{table}
For comparison purposes, we extend the analysis to new distributions, as the 
\pkg{sfaR} allows us to test different distribution.

\begin{Schunk}
\begin{Sinput}
R> ray <- sfacross(formula = log(tc/wf) ~ log(y) + I(1/2 * (log(y))^2) +
+   log(wl/wf) + log(wk/wf) + I(1/2 * (log(wl/wf))^2) + I(1/2 * (log(wk/wf))^2) +
+   I(log(wl/wf) * log(wk/wf)) + I(log(y) * log(wl/wf)) + I(log(y) * log(wk/wf)),
+   udist = 'rayleigh', uhet = ~ regu, data = utility, S = -1, method = 'bfgs')
R> ge <- sfacross(formula = log(tc/wf) ~ log(y) + I(1/2 * (log(y))^2) +
+   log(wl/wf) + log(wk/wf) + I(1/2 * (log(wl/wf))^2) + I(1/2 * (log(wk/wf))^2) +
+   I(log(wl/wf) * log(wk/wf)) + I(log(y) * log(wl/wf)) + I(log(y) * log(wk/wf)),
+   udist = 'genexponential', uhet = ~ regu, data = utility, S = -1, method = 'bfgs')
R> ga <- sfacross(formula = log(tc/wf) ~ log(y) + I(1/2 * (log(y))^2) +
+   log(wl/wf) + log(wk/wf) + I(1/2 * (log(wl/wf))^2) + I(1/2 * (log(wk/wf))^2) +
+   I(log(wl/wf) * log(wk/wf)) + I(log(y) * log(wl/wf)) + I(log(y) * log(wk/wf)),
+   udist = 'gamma', uhet = ~ regu, data = utility, S = -1, method = 'bfgs', Nsim = 300)
\end{Sinput}
\begin{Soutput}
Initialization of 300 Halton draws per observation ...
\end{Soutput}
\begin{Sinput}
R> we <- sfacross(formula = log(tc/wf) ~ log(y) + I(1/2 * (log(y))^2) +
+   log(wl/wf) + log(wk/wf) + I(1/2 * (log(wl/wf))^2) + I(1/2 * (log(wk/wf))^2) +
+   I(log(wl/wf) * log(wk/wf)) + I(log(y) * log(wl/wf)) + I(log(y) * log(wk/wf)),
+   udist = 'weibull', uhet = ~ regu, data = utility, S = -1, method = 'bfgs', Nsim = 300)
\end{Sinput}
\begin{Soutput}
Initialization of 300 Halton draws per observation ...
\end{Soutput}
\begin{Sinput}
R> lg <- sfacross(formula = log(tc/wf) ~ log(y) + I(1/2 * (log(y))^2) +
+   log(wl/wf) + log(wk/wf) + I(1/2 * (log(wl/wf))^2) + I(1/2 * (log(wk/wf))^2) +
+   I(log(wl/wf) * log(wk/wf)) + I(log(y) * log(wl/wf)) + I(log(y) * log(wk/wf)),
+   udist = 'lognormal', uhet = ~ regu, data = utility, S = -1, method = 'bfgs', Nsim = 300)
\end{Sinput}
\begin{Soutput}
Initialization of 300 Halton draws per observation ...
\end{Soutput}
\begin{Sinput}
R> ray <- sfacross(formula = log(tc/wf) ~ log(y) + I(1/2 * (log(y))^2) +
+   log(wl/wf) + log(wk/wf) + I(1/2 * (log(wl/wf))^2) + I(1/2 * (log(wk/wf))^2) +
+   I(log(wl/wf) * log(wk/wf)) + I(log(y) * log(wl/wf)) + I(log(y) * log(wk/wf)),
+   udist = 'tslaplace', uhet = ~ regu, data = utility, S = -1, method = 'bfgs')
R> ray <- sfacross(formula = log(tc/wf) ~ log(y) + I(1/2 * (log(y))^2) +
+   log(wl/wf) + log(wk/wf) + I(1/2 * (log(wl/wf))^2) + I(1/2 * (log(wk/wf))^2) +
+   I(log(wl/wf) * log(wk/wf)) + I(log(y) * log(wl/wf)) + I(log(y) * log(wk/wf)),
+   udist = 'uniform', uhet = ~ regu, data = utility, S = -1, method = 'bfgs')
\end{Sinput}
\end{Schunk}

put this table summary in appendix


using sfaR to compute productivity (us states data)


using sfaR to compute environmental efficiency

lcm as random parameters...

run for gamma in limdep to see!!!
maybe rayleigh also

if clear, exercice with fuglie data
restropo and tobon: maybe just give the link and they download it

%% -- Bibliography -------------------------------------------------------------
%% - References need to be provided in a .bib BibTeX database.
%% - All references should be made with \cite, \citet, \citep, \citealp etc.
%%   (and never hard-coded). See the FAQ for details.
%% - JSS-specific markup (\proglang, \pkg, \code) should be used in the .bib.
%% - Titles in the .bib should be in title case.
%% - DOIs should be included where available.

\bibliography{sfaR-bib}


\newpage

\begin{appendix}

\section{More on $u$ distributions}\label{app:moredens}

\setcounter{table}{0}
\renewcommand{\thetable}{\Alph{section}\arabic{table}}

%\begin{landscape}
\begin{table}[h]
%\setlength{\arrayrulewidth}{.01em}
\renewcommand{\arraystretch}{1.3}
\centering
\begin{adjustbox}{max width=1\textwidth}
\begin{tabular}{@{}cccc@{}}
\toprule
Distributions & $E\left[u\right]$ & $E\left[\exp{\left(-u\right)}\right]$ & $V\left[u\right]$ \\
\midrule
Half-Normal & $\sigma_u\sqrt{\frac{2}{\pi}}$ & $2\left[1-\Phi\left(\sigma_u\right)\right]\exp{\left(\frac{\sigma_u^2}{2}\right)}$ & 
$\frac{\pi-2}{\pi}\sigma_u^2$ \\[1em]
%\hdashline
Truncated-Normal & $\mu + \sigma_u\frac{\phi\left(\frac{\mu}{\sigma_u}\right)}{\Phi\left(\frac{\mu}{\sigma_u}\right)}$ & 
$\exp{\left(-\mu + \frac{1}{2}\sigma_u^2\right)}\frac{\Phi\left(\frac{\mu}{\sigma_u} - \sigma_u\right)}{\Phi\left(\frac{\mu}{\sigma_u}\right)}$ & 
$\sigma_u^2\left[1 - \frac{\mu}{\sigma_u} \frac{\phi\left(\frac{\mu}{\sigma_u}\right)}{\Phi\left(\frac{\mu}{\sigma_u}\right)} - \left(\frac{\phi\left(\frac{\mu}{\sigma_u}\right)}{\Phi\left(\frac{\mu}{\sigma_u}\right)}\right)^2\right]$\\[1em]
%\hdashline
Exponential & $\sigma_u$ & $\frac{1}{1+\sigma_u}$ & $\sigma_u^2$\\[1em]
%\hdashline
Rayleigh & $\sigma_u\sqrt{\frac{\pi}{2}}$ & $1 - \sigma_u\sqrt{2\pi}\left[1-\Phi\left(\sigma_u\right)\right]\exp{\left(\frac{\sigma_u^2}{2}\right)}$ &
$\frac{4-\pi}{2}\sigma_u^2$\\[1em]
%\hdashline
Gamma & $P\sigma_u$ & $\left(1+\sigma_u\right)^{-P}$ & $P\sigma_u^2$\\[1em]
%\hdashline
Log-Normal & $\exp{\left(\mu +\sigma_u^2/2\right)}$ & $\int_0^\infty \frac{\exp{\left(-u\right)}}{u\sigma_u}\phi\left(\frac{\ln{u}-\mu}{\sigma_u}\right)du$ & 
$\left[\exp{\left(\sigma_u^2\right)}-1\right]\exp{\left(2\mu +\sigma_u^2\right)}$\\[1em]
%\hdashline
Weibull & $\sigma_u\Gamma\left(1+1/k\right)$ & 
$\int_0^\infty\exp{\left(-u\right)}\frac{k}{\sigma_u}\left(\frac{u}{\sigma_u}\right)^{k-1}\exp{\left(-\left(u/\sigma_u\right)^{k}\right)}du$ & 
$\sigma_u^2\left[\Gamma\left(1+2/k\right)-\left(\Gamma\left(1+1/k\right)\right)^2\right]$\\[1em]
%\hdashline
Generalized Exponential & $\frac{3}{2}\sigma_u$ & $\frac{2}{\left(\sigma_u+1\right)\left(\sigma_u+2\right)}$ & $\frac{5}{4}\sigma_u^2$\\[1em]
%\hdashline
Truncated Skewed-Laplace & $\sigma_u\frac{1+4\lambda+2\lambda^2}{\left(1+\lambda\right)\left(1+2\lambda\right)}$ & 
$\frac{1+\lambda}{\left(2\lambda+1\right)}\left[\frac{2}{1+\sigma_u}-\frac{1}{1+\lambda + \sigma_u}\right]$ & 
$\sigma_u^2\frac{1+8\lambda+16\lambda^2+12\lambda^3+4\lambda^4}{\left(1+\lambda\right)^2\left(1+2\lambda\right)^2}$\\[1em]
%\hdashline
Uniform & $\frac{\theta}{2}$ & $\frac{1-\exp{\left(-\theta\right)}}{\theta}$ & $\exp{\left(W_u\right)}=\frac{\theta^2}{12}$\\[1em]
\bottomrule
\end{tabular}
\end{adjustbox}
\caption{List of distributions for $u$}
\label{table:momentdesn}
\end{table}
%\end{landscape}

\newpage

\section{Cross-sectional model tables} \label{app:sfcross}

\setcounter{table}{0}

\subsection{Densities of $f(\epsilon)$ for cross-sectional models}

%\newgeometry{left=2cm, right=2cm}
%\newgeometry{hmargin=2cm,vmargin=2cm}
%\begin{landscape}
%\pagestyle{lscape}
\begin{table}[h]
 \renewcommand{\arraystretch}{1.3}
% \setlength{\arrayrulewidth}{.01em}
 \centering
\begin{adjustbox}{max width=1\textwidth}
 \begin{tabular}{@{}ccc@{}}
\toprule
Distributions & $f(\epsilon)$ & Notes \\
\midrule
Half-Normal & $\frac{2\phi{\left(\frac{\epsilon}{\sqrt{\sigma_v^2 + \sigma_u^2}}
\right)}}{\sqrt{\sigma_u^2 + \sigma_v^2}}\Phi\left(\frac{\mu_*}{\sigma_*}
\right)$ & $\mu_{*}= \frac{-\epsilon S\sigma_u^2}{\sigma_u^2 + \sigma_v^2}$ and 
$\sigma_*^2=\frac{\sigma_u^2\sigma_v^2}{\sigma_u^2 + \sigma_v^2}$\\
%\hdashline 
Truncated-Normal & $\frac{\phi{\left(\frac{S\epsilon + \mu}{\sqrt{
\sigma_v^2 + \sigma_u^2}}\right)}}{\sqrt{\sigma_u^2 + \sigma_v^2}\Phi
\left(\frac{\mu}{\sigma_u}\right)}\Phi\left(\frac{\mu_*}{\sigma_*}\right)$ & 
$\mu_*= \frac{\mu\sigma_v^2 - \epsilon S\sigma_u^2}{\sigma_u^2 + \sigma_v^2}$ 
and $\sigma_*^2=\frac{\sigma_u^2\sigma_v^2}{\sigma_u^2 + \sigma_v^2}$ \\
%\hdashline 
Exponential & $\frac{1}{\sigma_u}\exp{\left(\frac{\sigma_v^2}{2\sigma_u^2}+
\frac{S\epsilon}{\sigma_u}\right)}\Phi\left(-\frac{\sigma_v}{\sigma_u} - 
\frac{S\epsilon}{\sigma_v}\right)$ & -\\
%\hdashline 
Rayleigh & $\frac{\exp{\left(\frac{\mu_*^2}{2\sigma_*^2} - \frac{\epsilon^2}{
2\sigma_v^2} \right)}}{\sigma_u^2\sigma_v}\sigma_* \left[\mu_*\Phi\left(
\frac{\mu_*}{\sigma_*}\right) + \sigma_*\phi\left(\frac{\mu_*}{\sigma_*}
\right)\right]$ & $\mu_{*}= \frac{-\epsilon S\sigma_u^2}{\sigma_u^2 + \sigma_v^2}$ 
and $\sigma_*^2=\frac{\sigma_u^2\sigma_v^2}{\sigma_u^2 + \sigma_v^2}$\\
%\hdashline
\multirow{2}{*}{Gamma} & \multirow{2}{*}{$\frac{\exp{\left(\frac{\sigma_v^2}{2\sigma_u^2} + \frac{S\epsilon}{
\sigma_u}\right)}}{\sigma_u^P\Gamma\left(P\right)}\Phi\left(-\frac{\sigma_v}{
\sigma_u} - \frac{S\epsilon}{\sigma_v}\right)\hat{h}(P-1, \epsilon)$} & 
$\hat{h}(P-1, \epsilon)= \frac{1}{R}\sum_{r = 1}^R \left[\mu + 
\sigma_v\Phi^{-1}\left(F_{r}+ \left(1-F_{r}\right)\Phi\left(-\frac{\mu}{
\sigma_v}\right)\right)\right]^{P-1}$\\
& &  $\mu = -\frac{\sigma_v^2}{\sigma_u} - S\epsilon_i$ and $F_{r}$ is pseudo/quasi random draw \\
%\hdashline 
Log-Normal & $\frac{1}{R}\sum_{r = 1}^R\frac{1}{\sigma_v}\phi\left(
\frac{\epsilon+Su_r}{\sigma_v}\right)$ & 
$u_r=\exp{\left(\mu +\sigma_u\Phi^{-1}\left(h_r\right)\right)}$ and $h_r$ is pseudo/quasi random draw\\
%\hdashline 
Weibull & $\frac{1}{R}\sum_{r = 1}^R\frac{1}{\sigma_v}\phi\left(\frac{
\epsilon+Su_r}{\sigma_v}\right)$ & $u_r = \sigma_u\left(-\ln{\left(1-h_r\right)}\right)^{1/k}$ and $h_r$ is pseudo/quasi random draw\\
%\hdashline 
\multirow{2}{*}{Generalized Exponential} & \multirow{2}{*}{$\frac{2}{\sigma_u}\exp{\left(A\right)}\Phi 
\left(a\right)-\frac{2}{\sigma_u}\exp{\left(B\right)}\Phi\left(b\right)$} & $A=\frac{S\epsilon}{\sigma_u}+\frac{\sigma_v^2}{2\sigma_u^2}$ and 
$B = \frac{2S\epsilon}{\sigma_u}+\frac{2\sigma_v^2}{\sigma_u^2}$\\
& & $a = -\frac{S\epsilon}{\sigma_v}-\frac{\sigma_v}{\sigma_u}$ and $b = -\frac{S\epsilon}{\sigma_v}- \frac{2\sigma_v}{\sigma_u}$ \\
%\hdashline 
\multirow{2}{*}{Truncated Skewed-Laplace} & \multirow{2}{*}{$\frac{1+\lambda}{\sigma_u\left(2\lambda+1\right)}
\left[2\exp{\left(A\right)}\Phi\left(a\right)-\exp{\left(B\right)}\Phi\left(b\right)\right]$} & 
$A = \left(\frac{\sigma_v^2}{2\sigma_u^2}+\frac{S\epsilon}{\sigma_u}\right)$ and 
$B = \left(\frac{\left(1+\lambda\right)^2\sigma_v^2}{2\sigma_u^2}+\frac{S\epsilon\left(1+\lambda\right)}{\sigma_u}\right)$\\
& & $a = -\frac{\sigma_v}{\sigma_u}-\frac{S\epsilon}{\sigma_v}$ and $b = -\frac{\left(1+\lambda\right)\sigma_v}{\sigma_u}-\frac{S\epsilon}{\sigma_v}$ \\
%\hdashline 
Uniform &  $\frac{1}{\theta}\left[\Phi\left(\frac{\theta + S\epsilon}{\sigma_v}
\right)-\Phi\left(\frac{S\epsilon}{\sigma_v}\right)\right]$ & -\\
\bottomrule
\end{tabular}
\end{adjustbox}
\caption{Densities of $f(\epsilon)$ for cross-section SF models}
\label{table:mlsfcross}
\end{table}
%\end{landscape}

%\restoregeometry

\newpage

\subsection{(In)efficiency estimates for cross-sectional models}

%\newgeometry{left=2cm, right=2cm}
%\begin{landscape}
%\pagestyle{lscape}
\begin{table}[h]
\begin{threeparttable}
\renewcommand{\arraystretch}{1.3}
%\setlength{\arrayrulewidth}{.01em}
\centering
\begin{adjustbox}{max width=1\textwidth}
%\small
\begin{tabular}{@{}cccc@{}}
\toprule
Densities & $E\left[u_i|\epsilon_i\right]$ & $E\left[\exp{\left(-u_i\right)|\epsilon_i}\right]$ & $E\left[\exp{\left(u_i\right)|\epsilon_i}\right]$\\
\midrule
Half-Normal & $\mu_{i*} + \sigma_*\frac{\phi\left(\frac{\mu_{i*}}{\sigma_*}\right)}{\Phi\left(\frac{\mu_{i*}}{\sigma_*}\right)}$ & 
$\exp{\left(-\mu_{i*} + \frac{1}{2}\sigma_*^2\right)} \frac{\Phi\left(\frac{\mu_{i*}}{\sigma_*}-\sigma_*\right)}{\Phi\left(\frac{\mu_{i*}}{\sigma_*}\right)}$ & 
$\exp{\left(\mu_{i*} + \frac{1}{2}\sigma_*^2\right)} \frac{\Phi\left(\frac{\mu_{i*}}{\sigma_*}+\sigma_*\right)}{\Phi\left(\frac{\mu_{i*}}{\sigma_*}\right)}$\\
%\hdashline
Truncated-Normal & $\mu_{i*} + \sigma_*\frac{\phi\left(\frac{\mu_{i*}}{\sigma_*}\right)}{\Phi\left(\frac{\mu_{i*}}{\sigma_*}\right)}$ & 
$\exp{\left(-\mu_{i*} + \frac{1}{2}\sigma_*^2\right)} \frac{\Phi\left(\frac{\mu_{i*}}{\sigma_*}-\sigma_*\right)}{\Phi\left(\frac{\mu_{i*}}{\sigma_*}\right)}$ & 
$\exp{\left(\mu_{i*} + \frac{1}{2}\sigma_*^2\right)} \frac{\Phi\left(\frac{\mu_{i*}}{\sigma_*}+\sigma_*\right)}{\Phi\left(\frac{\mu_{i*}}{\sigma_*}\right)}$ \\
%\hdashline
Exponential & $\mu_{i*} + \sigma_v\frac{\phi\left(\frac{\mu_{i*}}{\sigma_v}\right)}{\Phi\left(\frac{\mu_{i*}}{\sigma_v}\right)}$ & 
$\exp{\left(-\mu_{i*} + \frac{1}{2}\sigma_v^2\right)} \frac{\Phi\left(\frac{\mu_{i*}}{\sigma_v}-\sigma_v\right)}{\Phi\left(\frac{\mu_{i*}}{\sigma_v}\right)}$ & 
$\exp{\left(\mu_* + \frac{1}{2}\sigma_v^2\right)} \frac{\Phi\left(\frac{\mu_*}{\sigma_v}+\sigma_v\right)}{\Phi\left(\frac{\mu_*}{\sigma_v}\right)}$\\
%\hdashline
Rayleigh & $\frac{\left[\sigma_*\mu_{i*}\phi\left(\frac{\mu_{i*}}{\sigma_*}\right) + \left(\mu_{i*}^2+\sigma_*^2\right)\Phi\left(\frac{\mu_{i*}}{\sigma_*}\right)\right]}{\left[\mu_{i*}\Phi\left(\frac{\mu_{i*}}{\sigma_*}\right) + \sigma_*\phi\left(\frac{\mu_{i*}}{\sigma_*}\right)\right]}$ &
$\exp{\left(-\mu_{i*} +\frac{\sigma_*^2}{2}\right)}\frac{\left[\left(\mu_{i*}-\sigma_*^2\right)\Phi\left(\frac{\mu_{i*}}{\sigma_*} - \sigma_*\right) + \sigma_*\phi\left(\frac{\mu_{i*}}{\sigma_*} - \sigma_*\right)\right]}{\left[\mu_{i*}\Phi\left(\frac{\mu_{i*}}{\sigma_*}\right) + \sigma_*\phi\left(\frac{\mu_{i*}}{\sigma_*}\right)\right]}$ &
$\exp{\left(\mu_* +\frac{\sigma_*^2}{2}\right)}\frac{\left[\left(\mu_*+\sigma_*^2\right)\Phi\left(\frac{\mu_*}{\sigma_*} + \sigma_*\right) + \sigma_*\phi\left(\frac{\mu_*}{\sigma_*} + \sigma_*\right)\right]}{\left[\mu_*\Phi\left(\frac{\mu_*}{\sigma_*}\right) + \sigma_*\phi\left(\frac{\mu_*}{\sigma_*}\right)\right]}$\\
%\hdashline
Gamma & $\frac{h\left(P, \epsilon\right)}{h\left(P-1, \epsilon\right)}$ & $\frac{\exp{\left(\frac{\sigma_v^2}{\sigma_u} + S\epsilon + \frac{\sigma_v^2}{2}\right)}\Phi\left(-\frac{\sigma_v}{\sigma_u} - \frac{S\epsilon}{\sigma_v} - \sigma_v\right)  \hat{g}(P-1, \epsilon)}{\Phi\left(-\frac{\sigma_v}{\sigma_u} - \frac{S\epsilon}{\sigma_v}\right)\hat{h}(P-1, \epsilon)}$ & $\frac{\exp{\left(-\frac{\sigma_v^2}{\sigma_u} - S\epsilon + \frac{\sigma_v^2}{2}\right)}\Phi\left(-\frac{\sigma_v}{\sigma_u} - \frac{S\epsilon}{\sigma_v} + \sigma_v\right)  \hat{k}(P-1, \epsilon)}{\Phi\left(-\frac{\sigma_v}{\sigma_u} - \frac{S\epsilon}{\sigma_v}\right)\hat{h}(P-1, \epsilon)}$\\
%\hdashline
Log-Normal & $\frac{\frac{1}{\sigma_u\sigma_v}\int_0^\infty\phi\left(\frac{\ln{u}-\mu}{\sigma_u}\right)\phi\left(\frac{\epsilon_i+Su}{\sigma_v}\right)du}{\frac{1}{R}\sum_{r = 1}^R\frac{1}{\sigma_v}\phi\left(\frac{\epsilon_i+Su_{ir}}{\sigma_v}\right)}$ & 
$\frac{\frac{1}{\sigma_u\sigma_v}\int_0^\infty \frac{\exp{\left(-u\right)}}{u} \phi\left(\frac{\ln{u}-\mu}{\sigma_u}\right)\phi\left(\frac{\epsilon_i+Su}{\sigma_v}\right)du}{\frac{1}{R}\sum_{r = 1}^R\frac{1}{\sigma_v}\phi\left(\frac{\epsilon_i+Su_{ir}}{\sigma_v}\right)}$ & 
$\frac{\frac{1}{\sigma_u\sigma_v}\int_0^\infty \frac{\exp{\left(u\right)}}{u} \phi\left(\frac{\ln{u}-\mu}{\sigma_u}\right)\phi\left(\frac{\epsilon_i+Su}{\sigma_v}\right)du}{\frac{1}{R}\sum_{r = 1}^R\frac{1}{\sigma_v}\phi\left(\frac{\epsilon_i+Su_{ir}}{\sigma_v}\right)}$\\
%\hdashline
Weibull & $\frac{\frac{k}{\sigma_u^k\sigma_v}\int_0^\infty u^k\exp{\left(-\left(u/\sigma_u\right)^{k}\right)}\phi\left(\frac{\epsilon+Su}{\sigma_v}\right)du}{\frac{1}{R}\sum_{r = 1}^R\frac{1}{\sigma_v}\phi\left(\frac{\epsilon_i+Su_{ir}}{\sigma_v}\right)}$ & 
 $\frac{\frac{k}{\sigma_u\sigma_v}\int_0^\infty \exp{\left(-u_i\right)}\left(\frac{u}{\sigma_u}\right)^{k-1}\exp{\left(-\left(u/\sigma_u\right)^{k}\right)}\phi\left(\frac{\epsilon+Su}{\sigma_v}\right)du}{\frac{1}{R}\sum_{r = 1}^R\frac{1}{\sigma_v}\phi\left(\frac{\epsilon_i+Su_{ir}}{\sigma_v}\right)}$ & 
  $\frac{\frac{k}{\sigma_u\sigma_v}\int_0^\infty \exp{\left(u_i\right)}\left(\frac{u}{\sigma_u}\right)^{k-1}\exp{\left(-\left(u/\sigma_u\right)^{k}\right)}\phi\left(\frac{\epsilon+Su}{\sigma_v}\right)du}{\frac{1}{R}\sum_{r = 1}^R\frac{1}{\sigma_v}\phi\left(\frac{\epsilon_i+Su_{ir}}{\sigma_v}\right)}$\\
% \hdashline
Generalized Exponential & $\sigma_v \frac{\exp{\left(A\right)}\left[\phi\left(a\right) + a\Phi\left(a\right)\right]- \exp{\left(B\right)}\left[\phi\left(b\right)+b\Phi\left(b\right)\right]}{\exp{\left(A\right)}\Phi\left(a\right)-\exp{\left(B\right)}\Phi\left(b\right)}$ & 
$\frac{\exp{\left(A\right)}\exp{\left(-a\sigma_v+\frac{\sigma_v^2}{2}\right)}\Phi\left(a-\sigma_v\right)-\exp{\left(B\right)}\exp{\left(-b\sigma_v+\frac{\sigma_v^2}{2}\right)}\Phi\left(b-\sigma_v\right)}{\exp{\left(A\right)}\Phi\left(a\right)-\exp{\left(B\right)}\Phi\left(b\right)}$ & 
$\frac{\exp{\left(A\right)}\exp{\left(a\sigma_v+\frac{\sigma_v^2}{2}\right)}\Phi\left(a+\sigma_v\right)-\exp{\left(B\right)}\exp{\left(b\sigma_v+\frac{\sigma_v^2}{2}\right)}\Phi\left(b+\sigma_v\right)}{\exp{\left(A\right)}\Phi\left(a\right)-\exp{\left(B\right)}\Phi\left(b\right)}$\\
%\hdashline
Truncated Skewed-Laplace & $\sigma_v \frac{2\exp{\left(A\right)}\left[\phi\left(a\right) + a\Phi\left(a\right)\right]- \exp{\left(B\right)}\left[\phi\left(b\right)+b\Phi\left(b\right)\right]}{2\exp{\left(A\right)}\Phi\left(a\right)-\exp{\left(B\right)}\Phi\left(b\right)}$ & 
$\frac{2\exp{\left(A\right)}\exp{\left(-a\sigma_v+\frac{\sigma_v^2}{2}\right)}\Phi\left(a-\sigma_v\right)-\exp{\left(B\right)}\exp{\left(-b\sigma_v+\frac{\sigma_v^2}{2}\right)}\Phi\left(b-\sigma_v\right)}{2\exp{\left(A\right)}\Phi\left(a\right)-\exp{\left(B\right)}\Phi\left(b\right)}$ & 
$\frac{2\exp{\left(A\right)}\exp{\left(a\sigma_v+\frac{\sigma_v^2}{2}\right)}\Phi\left(a+\sigma_v\right)-\exp{\left(B\right)}\exp{\left(b\sigma_v+\frac{\sigma_v^2}{2}\right)}\Phi\left(b+\sigma_v\right)}{2\exp{\left(A\right)}\Phi\left(a\right)-\exp{\left(B\right)}\Phi\left(b\right)}$\\
%\hdashline
Uniform & $-\sigma_v\frac{\phi\left(\frac{\theta}{\sigma_v}+\frac{S\epsilon_i}{\sigma_v}\right)-\phi\left(\frac{S\epsilon_i}{\sigma_v}\right) }{\Phi\left(\frac{\theta}{\sigma_v}+\frac{S\epsilon_i}{\sigma_v}\right)-\Phi\left(\frac{S\epsilon_i}{\sigma_v}\right)} - S\epsilon_i$ & 
$\exp{\left(S\epsilon_i+\frac{\sigma_v^2}{2}\right)}\frac{\Phi\left(\frac{\theta}{\sigma_v}+\frac{S\epsilon_i}{\sigma_v}+\sigma_v\right)-\Phi\left(\frac{S\epsilon_i}{\sigma_v}+\sigma_v\right)}{\Phi\left(\frac{\theta}{\sigma_v}+\frac{S\epsilon_i}{\sigma_v}\right)-\Phi\left(\frac{S\epsilon_i}{\sigma_v}\right)}$ & 
$\exp{\left[-S\epsilon+\frac{\sigma_v^2}{2}\right]}\frac{\Phi\left(\frac{\theta}{\sigma_v}+\frac{S\epsilon}{\sigma_v}-\sigma_v\right)-\Phi\left(\frac{S\epsilon}{\sigma_v}-\sigma_v\right)}{\Phi\left(\frac{\theta}{\sigma_v}+\frac{S\epsilon}{\sigma_v}\right)-\Phi\left(\frac{S\epsilon}{\sigma_v}\right)}$\\
\bottomrule
\end{tabular}
\end{adjustbox}
\begin{tablenotes}
      \tiny
      \item Notes: For the exponential distribution we have $\mu_{i*} = -\left(\frac{\sigma_v^2}{\sigma_u} + S\epsilon\right)$. For the gamma distribution $\hat{g}(P-1, \epsilon)= \frac{1}{R}\sum_{r = 1}^R \left[\tilde{\mu} + 
\sigma_v\Phi^{-1}\left(F_{r}+ \left(1-F_{r}\right)\Phi\left(-\frac{\tilde{\mu}}{
\sigma_v}\right)\right)\right]^{P-1}$ with $\tilde{\mu} = \left(-\frac{\sigma_v^2}{\sigma_u} - S\epsilon - \sigma_v^2\right)$. 
      \item On the other hand $\hat{k}(P-1, \epsilon)= \frac{1}{R}\sum_{r = 1}^R \left[\tilde{\tilde{\mu}} + 
\sigma_v\Phi^{-1}\left(F_{r}+ \left(1-F_{r}\right)\Phi\left(-\frac{\tilde{\tilde{\mu}}}{
\sigma_v}\right)\right)\right]^{P-1}$ with $\tilde{\tilde{\mu}} = \left(-\frac{\sigma_v^2}{\sigma_u} - S\epsilon + \sigma_v^2\right)$. 
For all the other variables see notes in Table~\ref{table:mlsfcross}.
    \end{tablenotes}
\caption{Conditional (in)efficiencies for cross-section SF models}
\label{table:effcross}
\end{threeparttable}
\end{table}
%\end{landscape}
%\restoregeometry

\newpage

\section{Panel SF model tables} \label{app:sfpanel}

\setcounter{table}{0}
\subsection{Densities of $f(\epsilon)$ for time-invariant inefficiency models}



%\newgeometry{left=2cm, right=2cm}
%\begin{landscape}
%\pagestyle{lscape}
\begin{table}[h]
\renewcommand{\arraystretch}{1.3}
%\setlength{\arrayrulewidth}{.01em}
\centering
\begin{adjustbox}{max width=0.9\textwidth}
%\small
\begin{tabular}{@{}ccc@{}}
\toprule
Distributions & $f(\epsilon)$ & Notes \\
\midrule
Half-Normal & $\frac{2\sigma_*}{\sigma_u\sigma_v^T\left(2\pi\right)^{\frac{T}{2}}}\exp{\left[-\frac{1}{2}\left(-\frac{\mu_*^2}{\sigma_*^2} + \frac{\sum_{t=1}^T\epsilon_t^2}{\sigma_v^2}\right)\right]}\Phi\left(\frac{\mu_*}{\sigma_*}\right)$ & $\mu_{*}= \frac{-S\sigma_u^2\sum_{t=1}^T\epsilon_{t}}{T\sigma_u^2 + \sigma_v^2}$ and 
$\sigma_*^2=\frac{\sigma_u^2\sigma_v^2}{T\sigma_u^2 + \sigma_v^2}$\\
%\hdashline 
Truncated-Normal &  $\frac{\sigma_*}{\sigma_u\sigma_v^T\left(2\pi\right)^{\frac{T}{2}}\Phi\left(\frac{\mu}{\sigma_u}\right)}\exp{\left[-\frac{1}{2}\left(-\frac{\mu_*^2}{\sigma_*^2} + \frac{\sum_{t=1}^T\epsilon_t^2}{\sigma_v^2} + \frac{\mu^2}{\sigma_u^2}\right)\right]}\Phi\left(\frac{\mu_*}{\sigma_*}\right)$ & 
$\mu_{*}= \frac{\mu\sigma_v^2-S\sigma_u^2\sum_{t=1}^T\epsilon_{t}}{T\sigma_u^2+\sigma_v^2}$ and 
$\sigma_*^2=\frac{\sigma_u^2\sigma_v^2}{T\sigma_u^2 + \sigma_v^2}$\\
%\hdashline 
Exponential & $\frac{\sigma_*}{\sigma_u\sigma_v^T\left(2\pi\right)^{\frac{T-1}{2}}}\exp{\left[-\frac{1}{2}\left(\frac{\sum_{t=1}^T\epsilon_t^2}{\sigma_v^2}-\frac{\mu_*^2}{\sigma_*^2}\right)\right]}\Phi\left(\frac{\mu_*}{\sigma_*}\right)$ & 
$\mu_* = -\left(\frac{\sigma_v^2}{T\sigma_u}+\frac{S\sum_{t=1}^T\epsilon_t}{T}\right)$ and 
$\sigma_*^2 = \frac{\sigma_v^2}{T}$\\
%\hdashline 
Rayleigh & $\frac{\sigma_*}{\sigma_u^2\sigma_v^T\left(2\pi\right)^{\frac{T-1}{2}}}\exp{\left[-\frac{1}{2}\left(\frac{\sum_{t=1}^T\epsilon_t^2}{\sigma_v^2}-\frac{\mu_*^2}{\sigma_*^2}\right)\right]}\left[\mu_*\Phi\left(\frac{\mu_*}{\sigma_*}\right)+\sigma_*\phi\left(\frac{\mu_*}{\sigma_*}\right)\right]$ & 
$\mu_* = -\frac{S\sigma_u^2\sum_{t=1}^T\epsilon_t}{T\sigma_u^2+\sigma_v^2}$ and $\sigma_*^2 = \frac{\sigma_u^2\sigma_v^2}{T\sigma_u^2+\sigma_v^2}$\\
%\hdashline
\multirow{2}{*}{Gamma} & \multirow{2}{*}{$\frac{\sigma_*}{\sigma_u^P\sigma_v^{T}\left(2\pi\right)^{\frac{T-1}{2}}\Gamma(P)}\exp{\left[-\frac{1}{2}\left(\frac{\sum_{t=1}^T\epsilon_t^2}{\sigma_v^2}-\frac{\mu_*^2}{\sigma_*^2}\right)\right]}
\Phi\left(\frac{\mu_*}{\sigma_*}\right) \hat{h}(P-1, \epsilon)$} & 
$h(P-1, \epsilon)=\frac{1}{R}\sum_{r = 1}^R \left(\mu_* + \sigma_*\Phi^{-1}\left[PL + F_r \times \left(1 - PL\right)\right]\right)^{P-1}$ and 
$\mu_* = -\left(\frac{\sigma_v^2}{T\sigma_u}+\frac{S\sum_{t=1}^T\epsilon_t}{T}\right)$\\
& & $\sigma_*^2 = \frac{\sigma_v^2}{T}$ and $F_r$ is pseudo/quasi random draw\\
%\hdashline 
Log-Normal & $\frac{1}{R}\sum_{r = 1}^R\frac{1}{\left(2\pi\right)^{T/2}\sigma_v^T}\exp{\left(-\frac{\sum_{t=1}^T\left(\epsilon_t+Su_r\right)^2}{2\sigma_v^2}\right)}$ & 
$u_r=\exp{\left(\mu +\sigma_u\Phi^{-1}\left(h_r\right)\right)}$ and $h_r$ is pseudo/quasi random draw\\
%\hdashline 
Weibull & $\frac{1}{R}\sum_{r = 1}^R\frac{1}{\left(2\pi\right)^{T/2}\sigma_v^T}\exp{\left(-\frac{\sum_{t=1}^T\left(\epsilon_t+Su_r\right)^2}{2\sigma_v^2}\right)}$ & 
$u_r = \sigma_u\left(-\ln{\left(1-h_r\right)}\right)^{1/k}$ and $h_r$ is pseudo/quasi random draw\\
%\hdashline 
\multirow{2}{*}{Generalized Exponential} & \multirow{2}{*}{$\frac{2\sigma_*}{\sigma_u\sigma_v^T\left(2\pi\right)^{\frac{T-1}{2}}}\left\{\exp{\left[-\frac{1}{2}\left(\frac{\sum_{t=1}^T\epsilon_t^2}{\sigma_v^2}-\frac{\mu_*^2}{\sigma_*^2}\right)\right]}\Phi\left(\frac{\mu_*}{\sigma_*}\right)-\exp{\left[-\frac{1}{2}\left(\frac{\sum_{t=1}^T\epsilon_t^2}{\sigma_v^2}-\frac{\mu_*'^2}{\sigma_*^2}\right)\right]}\Phi\left(\frac{\mu_*'}{\sigma_*}\right)\right\}$} & $\mu_* = -\left(\frac{\sigma_v^2}{T\sigma_u}+\frac{S\sum_{t=1}^T\epsilon_t}{T}\right)$ and 
$\mu_*' = -\left(\frac{2\sigma_v^2}{T\sigma_u}+\frac{S\sum_{t=1}^T\epsilon_t}{T}\right)$\\
& & $\sigma_*^2 = \frac{\sigma_v^2}{T}$\\
%\hdashline 
\multirow{2}{*}{Truncated Skewed-Laplace} & \multirow{2}{*}{$\frac{\left(1+\lambda\right)\sigma_*}{\sigma_u\sigma_v^T\left(2\pi\right)^{\frac{T-1}{2}}\left(2\lambda+1\right)}\left\{2\exp{\left[-\frac{1}{2}\left(\frac{\sum_{t=1}^T\epsilon_t^2}{\sigma_v^2}-\frac{\mu_*^2}{\sigma_*^2}\right)\right]}\Phi\left(\frac{\mu_*}{\sigma_*}\right)-\exp{\left[-\frac{1}{2}\left(\frac{\sum_{t=1}^T\epsilon_t^2}{\sigma_v^2}-\frac{\mu_*'^2}{\sigma_*^2}\right)\right]}\Phi\left(\frac{\mu_*'}{\sigma_*}\right)\right\}$} & 
$\mu_* = -\left(\frac{\sigma_v^2}{T\sigma_u}+\frac{S\sum_{t=1}^T\epsilon_t}{T}\right)$ and $\mu_*' = -\left(\frac{\left(1+\lambda\right)\sigma_v^2}{T\sigma_u}+\frac{S\sum_{t=1}^T\epsilon_t}{T}\right)$\\
& & $\sigma_*^2 = \frac{\sigma_v^2}{T}$ \\
%\hdashline 
Uniform & $\frac{\sigma_*}{\theta\sigma_v^T\left(2\pi\right)^{\frac{T-1}{2}}}\exp{\left[-\frac{1}{2}\left(\frac{\sum_{t=1}^T\epsilon_t^2}{\sigma_v^2}-\frac{\mu_*^2}{\sigma_*^2}\right)\right]}\left[\Phi\left(\frac{\theta-\mu_*}{\sigma_*}\right)-\Phi\left(-\frac{\mu_*}{\sigma_*}\right)\right]$ & 
$\mu_* = -\frac{S\sum_{t=1}^T\epsilon_t}{T}$ and $\sigma_*^2 = \frac{\sigma_v^2}{T}$\\
\bottomrule
\end{tabular}
\end{adjustbox}
\caption{Densities of $f(\epsilon)$ for time-invariant inefficiency models}
\label{table:mlpl81}
\end{table}
%\end{landscape}
%\restoregeometry

\newpage

\subsection{Densities of $f(\epsilon)$ for time-varying inefficiency models}

%\newgeometry{left=2cm, right=2cm}
%\begin{landscape}
\pagestyle{lscape}
\begin{table}[h]
\renewcommand{\arraystretch}{1.3}
%\setlength{\arrayrulewidth}{.01em}
\centering
\begin{adjustbox}{max width=0.9\textwidth}
%\small
\begin{tabular}{@{}ccc@{}}
\toprule
Distributions & $f(\epsilon)$ & Notes \\
\midrule
Half-Normal & $\frac{2\sigma_*}{\sigma_u\sigma_v^T\left(2\pi\right)^{\frac{T}{2}}}\exp{\left[-\frac{1}{2}\left(-\frac{\mu_*^2}{\sigma_*^2} + \frac{\sum_{t=1}^T\epsilon_t^2}{\sigma_v^2}\right)\right]}\Phi\left(\frac{\mu_*}{\sigma_*}\right)$ & $\mu_{*}= \frac{-S\sigma_u^2\sum_{t=1}^TG(t)\epsilon_{t}}{\sigma_u^2\sum_{t=1}^TG(t)^2 + \sigma_v^2}$ and 
$\sigma_*^2=\frac{\sigma_u^2\sigma_v^2}{\sigma_u^2\sum_{t=1}^TG(t)^2 + \sigma_v^2}$\\
%\hdashline 
Truncated-Normal &  $\frac{\sigma_*}{\sigma_u\sigma_v^T\left(2\pi\right)^{\frac{T}{2}}\Phi\left(\frac{\mu}{\sigma_u}\right)}\exp{\left[-\frac{1}{2}\left(-\frac{\mu_*^2}{\sigma_*^2} + \frac{\sum_{t=1}^T\epsilon_t^2}{\sigma_v^2} + \frac{\mu^2}{\sigma_u^2}\right)\right]}\Phi\left(\frac{\mu_*}{\sigma_*}\right)$ & 
$\mu_{*}= \frac{\mu\sigma_v^2-S\sigma_u^2\sum_{t=1}^TG(t)\epsilon_t}{\sigma_u^2\sum_{t=1}G(t)^2+\sigma_v^2}$ and 
$\sigma_*^2=\frac{\sigma_u^2\sigma_v^2}{\sigma_u^2\sum_{t=1}^TG(t)^2 + \sigma_v^2}$\\
%\hdashline 
Exponential & $\frac{\sigma_*}{\sigma_u\sigma_v^T\left(2\pi\right)^{\frac{T-1}{2}}}\exp{\left[-\frac{1}{2}\left(\frac{\sum_{t=1}^T\epsilon_t^2}{\sigma_v^2}-\frac{\mu_*^2}{\sigma_*^2}\right)\right]}\Phi\left(\frac{\mu_*}{\sigma_*}\right)$ & 
$\mu_* = -\left(\frac{\sigma_v^2}{\sigma_u\sum_{t=1}G(t)^2}+\frac{S\sum_{t=1}^TG(t)\epsilon_t}{\sum_{t=1}G(t)^2}\right)$ and 
$\sigma_*^2 = \frac{\sigma_v^2}{\sum_{t=1}G(t)^2}$\\
%\hdashline 
Rayleigh & $\frac{\sigma_*}{\sigma_u^2\sigma_v^T\left(2\pi\right)^{\frac{T-1}{2}}}\exp{\left[-\frac{1}{2}\left(\frac{\sum_{t=1}^T\epsilon_t^2}{\sigma_v^2}-\frac{\mu_*^2}{\sigma_*^2}\right)\right]}\left[\mu_*\Phi\left(\frac{\mu_*}{\sigma_*}\right)+\sigma_*\phi\left(\frac{\mu_*}{\sigma_*}\right)\right]$ & 
$\mu_* = -\frac{S\sigma_u^2\sum_{t=1}^TG(t)\epsilon_t}{\sigma_u^2\sum_{t=1}G(t)^2+\sigma_v^2}$ and $\sigma_*^2 = \frac{\sigma_u^2\sigma_v^2}{\sigma_u^2\sum_{t=1}G(t)^2+\sigma_v^2}$\\
%\hdashline
\multirow{2}{*}{Gamma} & \multirow{2}{*}{$\frac{\sigma_*}{\sigma_u^P\sigma_v^{T}\left(2\pi\right)^{\frac{T-1}{2}}\Gamma(P)}\exp{\left[-\frac{1}{2}\left(\frac{\sum_{t=1}^T\epsilon_t^2}{\sigma_v^2}-\frac{\mu_*^2}{\sigma_*^2}\right)\right]}
\Phi\left(\frac{\mu_*}{\sigma_*}\right) \hat{h}(P-1, \epsilon)$} & 
$h(P-1, \epsilon)=\frac{1}{R}\sum_{r = 1}^R \left(\mu_* + \sigma_*\Phi^{-1}\left[PL + F_r \times \left(1 - PL\right)\right]\right)^{P-1}$ and 
$\mu_* = -\left(\frac{\sigma_v^2}{\sigma_u\sum_{t=1}G(t)^2}+\frac{S\sum_{t=1}^TG(t)\epsilon_t}{\sum_{t=1}G(t)^2}\right)$\\
& & $\sigma_*^2 = \frac{\sigma_v^2}{\sum_{t=1}G(t)^2}$ and $F_r$ is pseudo/quasi random draw\\
%\hdashline 
Log-Normal & $\frac{1}{R}\sum_{r = 1}^R\frac{1}{\left(2\pi\right)^{T/2}\sigma_v^T}\exp{\left(-\frac{\sum_{t=1}^T\left(\epsilon_t+SG(t)u_r\right)^2}{2\sigma_v^2}\right)}$ & 
$u_r=\exp{\left(\mu +\sigma_u\Phi^{-1}\left(h_r\right)\right)}$ and $h_r$ is pseudo/quasi random draw\\
%\hdashline 
Weibull & $\frac{1}{R}\sum_{r = 1}^R\frac{1}{\left(2\pi\right)^{T/2}\sigma_v^T}\exp{\left(-\frac{\sum_{t=1}^T\left(\epsilon_t+SG(t)u_r\right)^2}{2\sigma_v^2}\right)}$ & 
$u_r = \sigma_u\left(-\ln{\left(1-h_r\right)}\right)^{1/k}$ and $h_r$ is pseudo/quasi random draw\\
%\hdashline 
\multirow{2}{*}{Generalized Exponential} & \multirow{2}{*}{$\frac{2\sigma_*}{\sigma_u\sigma_v^T\left(2\pi\right)^{\frac{T-1}{2}}}\left\{\exp{\left[-\frac{1}{2}\left(\frac{\sum_{t=1}^T\epsilon_t^2}{\sigma_v^2}-\frac{\mu_*^2}{\sigma_*^2}\right)\right]}\Phi\left(\frac{\mu_*}{\sigma_*}\right)-\exp{\left[-\frac{1}{2}\left(\frac{\sum_{t=1}^T\epsilon_t^2}{\sigma_v^2}-\frac{\mu_*'^2}{\sigma_*^2}\right)\right]}\Phi\left(\frac{\mu_*'}{\sigma_*}\right)\right\}$} & $\mu_* = -\left(\frac{\sigma_v^2}{\sigma_u\sum_{t=1}G(t)^2}+\frac{S\sum_{t=1}^TG(t)\epsilon_t}{\sum_{t=1}G(t)^2}\right)$ and 
$\mu_*' = -\left(\frac{2\sigma_v^2}{\sigma_u\sum_{t=1}G(t)^2}+\frac{S\sum_{t=1}^TG(t)\epsilon_t}{\sum_{t=1}G(t)^2}\right)$\\
& & $\sigma_*^2 = \frac{\sigma_v^2}{\sum_{t=1}G(t)^2}$\\
%\hdashline 
\multirow{2}{*}{Truncated Skewed-Laplace} & \multirow{2}{*}{$\frac{\left(1+\lambda\right)\sigma_*}{\sigma_u\sigma_v^T\left(2\pi\right)^{\frac{T-1}{2}}\left(2\lambda+1\right)}\left\{2\exp{\left[-\frac{1}{2}\left(\frac{\sum_{t=1}^T\epsilon_t^2}{\sigma_v^2}-\frac{\mu_*^2}{\sigma_*^2}\right)\right]}\Phi\left(\frac{\mu_*}{\sigma_*}\right)-\exp{\left[-\frac{1}{2}\left(\frac{\sum_{t=1}^T\epsilon_t^2}{\sigma_v^2}-\frac{\mu_*'^2}{\sigma_*^2}\right)\right]}\Phi\left(\frac{\mu_*'}{\sigma_*}\right)\right\}$} & 
$\mu_* = -\left(\frac{\sigma_v^2}{\sigma_u\sum_{t=1}G(t)^2}+\frac{S\sum_{t=1}^TG(t)\epsilon_t}{\sum_{t=1}G(t)^2}\right)$ and $\mu_*' = -\left(\frac{\left(1+\lambda\right)\sigma_v^2}{\sigma_u\sum_{t=1}G(t)^2}+\frac{S\sum_{t=1}^TG(t)\epsilon_t}{\sum_{t=1}G(t)^2}\right)$\\
& & $\sigma_*^2 = \frac{\sigma_v^2}{\sum_{t=1}G(t)^2}$ \\
%\hdashline 
Uniform & $\frac{\sigma_*}{\theta\sigma_v^T\left(2\pi\right)^{\frac{T-1}{2}}}\exp{\left[-\frac{1}{2}\left(\frac{\sum_{t=1}^T\epsilon_t^2}{\sigma_v^2}-\frac{\mu_*^2}{\sigma_*^2}\right)\right]}\left[\Phi\left(\frac{\theta-\mu_*}{\sigma_*}\right)-\Phi\left(-\frac{\mu_*}{\sigma_*}\right)\right]$ & 
$\mu_* = -\frac{S\sum_{t=1}^TG(t)\epsilon_t}{\sum_{t=1}G(t)^2}$ and $\sigma_*^2 = \frac{\sigma_v^2}{\sum_{t=1}G(t)^2}$\\
\bottomrule
\end{tabular}
\end{adjustbox}
\caption{Densities of $f(\epsilon)$ for time-varying inefficiency models}
\label{table:mlgzit}
\end{table}
%\end{landscape}
%\restoregeometry

\newpage

\subsection{(In)efficiencies for time-varying inefficiency models}

%\newgeometry{left=2cm, right=2cm}
%\begin{landscape}
%\pagestyle{lscape}
\begin{table}
\renewcommand{\arraystretch}{1.3}
%\setlength{\arrayrulewidth}{.01em}
\centering
\begin{adjustbox}{max width=0.9\textwidth}
%\small
\begin{tabular}{@{}cccc@{}}
\toprule
Densities & $E\left[G(t)u_i|\bm{\epsilon}_i\right]$ & $E\left[\exp{\left(-G(t)u_i\right)|\epsilon_i}\right]$ & $E\left[\exp{\left(G(t)u_i\right)|\epsilon_i}\right]$\\
Half-Normal & & $\exp{\left[\frac{1}{2}\sigma_*^2G(t)^2-\mu_*G(t)\right]\frac{\Phi\left(\frac{\mu_*}{\sigma_*}-G(t)\sigma_*\right)}{\Phi\left(\frac{\mu_*}{\sigma_*}\right)}}$
&  $\exp{\left[\frac{1}{2}\sigma_*^2G(t)^2+\mu_*G(t)\right]\frac{\Phi\left(\frac{\mu_*}{\sigma_*}+G(t)\sigma_*\right)}{\Phi\left(\frac{\mu_*}{\sigma_*}\right)}}$ \\
%\hdashline 
Truncated-Normal & & $\exp{\left[\frac{1}{2}\sigma_*^2G(t)^2-\mu_*G(t)\right]\frac{\Phi\left(\frac{\mu_*}{\sigma_*}-G(t)\sigma_*\right)}{\Phi\left(\frac{\mu_*}{\sigma_*}\right)}}$
&  $\exp{\left[\frac{1}{2}\sigma_*^2G(t)^2+\mu_*G(t)\right]\frac{\Phi\left(\frac{\mu_*}{\sigma_*}+G(t)\sigma_*\right)}{\Phi\left(\frac{\mu_*}{\sigma_*}\right)}}$  \\
%\hdashline 
Exponential & $\mu_{i*} + \sigma_v\frac{\phi\left(\frac{\mu_{i*}}{\sigma_v}\right)}{\Phi\left(\frac{\mu_{i*}}{\sigma_*}\right)}$ & 
$\exp{\left(-G(t)\mu_{i*} + \frac{1}{2}G(t)^2\sigma_v^2\right)} \frac{\Phi\left(\frac{\mu_{i*}}{\sigma_v}-G(t)\sigma_v\right)}{\Phi\left(\frac{\mu_{i*}}{\sigma_v}\right)}$ & 
$\exp{\left(G(t)\mu_* + \frac{1}{2}G(t)^2\sigma_v^2\right)} \frac{\Phi\left(\frac{\mu_*}{\sigma_v}+G(t)\sigma_v\right)}{\Phi\left(\frac{\mu_*}{\sigma_v}\right)}$\\
%\hdashline
Rayleigh & $\frac{\left[\sigma_*\mu_{i*}\phi\left(\frac{\mu_{i*}}{\sigma_*}\right) + \left(\mu_{i*}^2+\sigma_*^2\right)\Phi\left(\frac{\mu_{i*}}{\sigma_*}\right)\right]}{\left[\mu_{i*}\Phi\left(\frac{\mu_{i*}}{\sigma_*}\right) + \sigma_*\phi\left(\frac{\mu_{i*}}{\sigma_*}\right)\right]}$ &
$\exp{\left(-G(t)\mu_{i*} +\frac{G(t)^2\sigma_*^2}{2}\right)}\frac{\left[\left(G(t)\mu_{i*}-G(t)^2\sigma_*^2\right)\Phi\left(\frac{\mu_{i*}}{\sigma_*} - G(t)\sigma_*\right) + G(t)\sigma_*\phi\left(\frac{\mu_{i*}}{\sigma_*} - G(t)\sigma_*\right)\right]}{\left[\mu_{i*}\Phi\left(\frac{\mu_{i*}}{\sigma_*}\right) + \sigma_*\phi\left(\frac{\mu_{i*}}{\sigma_*}\right)\right]}$ &
$\exp{\left(G(t)\mu_* +\frac{G(t)^2\sigma_*^2}{2}\right)}\frac{\left[\left(G(t)\mu_*+G(t)^2\sigma_*^2\right)\Phi\left(\frac{\mu_*}{\sigma_*} + \sigma_*\right) + \sigma_*\phi\left(\frac{\mu_*}{\sigma_*} + \sigma_*\right)\right]}{\left[\mu_*\Phi\left(\frac{\mu_*}{\sigma_*}\right) + \sigma_*\phi\left(\frac{\mu_*}{\sigma_*}\right)\right]}$\\
%\hdashline
Gamma & $\frac{h\left(P, \epsilon\right)}{h\left(P-1, \epsilon\right)}$ & $\frac{\exp{\left(\frac{\sigma_v^2}{\sigma_u} + S\epsilon + \frac{\sigma_v^2}{2}\right)}\Phi\left(-\frac{\sigma_v}{\sigma_u} - \frac{S\epsilon}{\sigma_v} - \sigma_v\right)  \hat{g}(P-1, \epsilon)}{\Phi\left(-\frac{\sigma_v}{\sigma_u} - \frac{S\epsilon}{\sigma_v}\right)\hat{h}(P-1, \epsilon)}$ & $\frac{\exp{\left(-\frac{\sigma_v^2}{\sigma_u} - S\epsilon + \frac{\sigma_v^2}{2}\right)}\Phi\left(-\frac{\sigma_v}{\sigma_u} - \frac{S\epsilon}{\sigma_v} + \sigma_v\right)  \hat{k}(P-1, \epsilon)}{\Phi\left(-\frac{\sigma_v}{\sigma_u} - \frac{S\epsilon}{\sigma_v}\right)\hat{h}(P-1, \epsilon)}$\\
%\hdashline
Log-Normal & $\frac{\frac{1}{\sigma_u\sigma_v}\int_0^\infty\phi\left(\frac{\ln{u}-\mu}{\sigma_u}\right)\phi\left(\frac{\epsilon_i+Su}{\sigma_v}\right)du}{\frac{1}{R}\sum_{r = 1}^R\frac{1}{\sigma_v}\phi\left(\frac{\epsilon_i+Su_{ir}}{\sigma_v}\right)}$ & 
$\frac{\frac{1}{\sigma_u\sigma_v}\int_0^\infty \frac{\exp{\left(-u\right)}}{u} \phi\left(\frac{\ln{u}-\mu}{\sigma_u}\right)\phi\left(\frac{\epsilon_i+Su}{\sigma_v}\right)du}{\frac{1}{R}\sum_{r = 1}^R\frac{1}{\sigma_v}\phi\left(\frac{\epsilon_i+Su_{ir}}{\sigma_v}\right)}$ & 
$\frac{\frac{1}{\sigma_u\sigma_v}\int_0^\infty \frac{\exp{\left(u\right)}}{u} \phi\left(\frac{\ln{u}-\mu}{\sigma_u}\right)\phi\left(\frac{\epsilon_i+Su}{\sigma_v}\right)du}{\frac{1}{R}\sum_{r = 1}^R\frac{1}{\sigma_v}\phi\left(\frac{\epsilon_i+Su_{ir}}{\sigma_v}\right)}$\\
%\hdashline
Weibull & $\frac{\frac{k}{\sigma_u^k\sigma_v}\int_0^\infty u^k\exp{\left(-\left(u/\sigma_u\right)^{k}\right)}\phi\left(\frac{\epsilon+Su}{\sigma_v}\right)du}{\frac{1}{R}\sum_{r = 1}^R\frac{1}{\sigma_v}\phi\left(\frac{\epsilon_i+Su_{ir}}{\sigma_v}\right)}$ & 
 $\frac{\frac{k}{\sigma_u\sigma_v}\int_0^\infty \exp{\left(-u_i\right)}\left(\frac{u}{\sigma_u}\right)^{k-1}\exp{\left(-\left(u/\sigma_u\right)^{k}\right)}\phi\left(\frac{\epsilon+Su}{\sigma_v}\right)du}{\frac{1}{R}\sum_{r = 1}^R\frac{1}{\sigma_v}\phi\left(\frac{\epsilon_i+Su_{ir}}{\sigma_v}\right)}$ & 
  $\frac{\frac{k}{\sigma_u\sigma_v}\int_0^\infty \exp{\left(u_i\right)}\left(\frac{u}{\sigma_u}\right)^{k-1}\exp{\left(-\left(u/\sigma_u\right)^{k}\right)}\phi\left(\frac{\epsilon+Su}{\sigma_v}\right)du}{\frac{1}{R}\sum_{r = 1}^R\frac{1}{\sigma_v}\phi\left(\frac{\epsilon_i+Su_{ir}}{\sigma_v}\right)}$\\
 % \hdashline
Generalized Exponential & $\sigma_v \frac{\exp{\left(A\right)}\left[\phi\left(a\right) + a\Phi\left(a\right)\right]- \exp{\left(B\right)}\left[\phi\left(b\right)+b\Phi\left(b\right)\right]}{\exp{\left(A\right)}\Phi\left(a\right)-\exp{\left(B\right)}\Phi\left(b\right)}$ & 
$\frac{\exp{\left(A\right)}\exp{\left(-a\sigma_v+\frac{\sigma_v^2}{2}\right)}\Phi\left(a-\sigma_v\right)-\exp{\left(B\right)}\exp{\left(-b\sigma_v+\frac{\sigma_v^2}{2}\right)}\Phi\left(b-\sigma_v\right)}{\exp{\left(A\right)}\Phi\left(a\right)-\exp{\left(B\right)}\Phi\left(b\right)}$ & 
$\frac{\exp{\left(A\right)}\exp{\left(a\sigma_v+\frac{\sigma_v^2}{2}\right)}\Phi\left(a+\sigma_v\right)-\exp{\left(B\right)}\exp{\left(b\sigma_v+\frac{\sigma_v^2}{2}\right)}\Phi\left(b+\sigma_v\right)}{\exp{\left(A\right)}\Phi\left(a\right)-\exp{\left(B\right)}\Phi\left(b\right)}$\\
%\hdashline
Truncated Skewed-Laplace & $\sigma_v \frac{2\exp{\left(A\right)}\left[\phi\left(a\right) + a\Phi\left(a\right)\right]- \exp{\left(B\right)}\left[\phi\left(b\right)+b\Phi\left(b\right)\right]}{2\exp{\left(A\right)}\Phi\left(a\right)-\exp{\left(B\right)}\Phi\left(b\right)}$ & 
$\frac{2\exp{\left(A\right)}\exp{\left(-a\sigma_v+\frac{\sigma_v^2}{2}\right)}\Phi\left(a-\sigma_v\right)-\exp{\left(B\right)}\exp{\left(-b\sigma_v+\frac{\sigma_v^2}{2}\right)}\Phi\left(b-\sigma_v\right)}{2\exp{\left(A\right)}\Phi\left(a\right)-\exp{\left(B\right)}\Phi\left(b\right)}$ & 
$\frac{2\exp{\left(A\right)}\exp{\left(a\sigma_v+\frac{\sigma_v^2}{2}\right)}\Phi\left(a+\sigma_v\right)-\exp{\left(B\right)}\exp{\left(b\sigma_v+\frac{\sigma_v^2}{2}\right)}\Phi\left(b+\sigma_v\right)}{2\exp{\left(A\right)}\Phi\left(a\right)-\exp{\left(B\right)}\Phi\left(b\right)}$\\
%\hdashline
Uniform & $-\sigma_v\frac{\phi\left(\frac{\theta}{\sigma_v}+\frac{S\epsilon_i}{\sigma_v}\right)-\phi\left(\frac{S\epsilon_i}{\sigma_v}\right) }{\Phi\left(\frac{\theta}{\sigma_v}+\frac{S\epsilon_i}{\sigma_v}\right)-\Phi\left(\frac{S\epsilon_i}{\sigma_v}\right)} - S\epsilon_i$ & 
$\exp{\left(S\epsilon_i+\frac{\sigma_v^2}{2}\right)}\frac{\Phi\left(\frac{\theta}{\sigma_v}+\frac{S\epsilon_i}{\sigma_v}+\sigma_v\right)-\Phi\left(\frac{S\epsilon_i}{\sigma_v}+\sigma_v\right)}{\Phi\left(\frac{\theta}{\sigma_v}+\frac{S\epsilon_i}{\sigma_v}\right)-\Phi\left(\frac{S\epsilon_i}{\sigma_v}\right)}$ & 
$\exp{\left[-S\epsilon+\frac{\sigma_v^2}{2}\right]}\frac{\Phi\left(\frac{\theta}{\sigma_v}+\frac{S\epsilon}{\sigma_v}-\sigma_v\right)-\Phi\left(\frac{S\epsilon}{\sigma_v}-\sigma_v\right)}{\Phi\left(\frac{\theta}{\sigma_v}+\frac{S\epsilon}{\sigma_v}\right)-\Phi\left(\frac{S\epsilon}{\sigma_v}\right)}$\\
\bottomrule
\end{tabular}
\end{adjustbox}
\caption{Conditional (in)efficiencies for time-varying inefficiency models}
\label{table:effpanel}
\end{table}
%\end{landscape}
%\restoregeometry


\end{appendix}

\end{document}
